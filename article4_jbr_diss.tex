
% https://www.gnu.org/software/emacs/manual/html_node/emacs/Fill-Commands.html
%https://academic.oup.com/journals/pages/authors/latex_files
%latexmk -pdf -pvc -auxdir=/Users/larshovdanmolden/Documents/Output/  article2_mancap.tex

\documentclass[review,fleqn]{elsarticle}\usepackage[]{graphicx}\usepackage[]{color}
% maxwidth is the original width if it is less than linewidth
% otherwise use linewidth (to make sure the graphics do not exceed the margin)
\makeatletter
\def\maxwidth{ %
  \ifdim\Gin@nat@width>\linewidth
    \linewidth
  \else
    \Gin@nat@width
  \fi
}
\makeatother

\definecolor{fgcolor}{rgb}{0.345, 0.345, 0.345}
\newcommand{\hlnum}[1]{\textcolor[rgb]{0.686,0.059,0.569}{#1}}%
\newcommand{\hlstr}[1]{\textcolor[rgb]{0.192,0.494,0.8}{#1}}%
\newcommand{\hlcom}[1]{\textcolor[rgb]{0.678,0.584,0.686}{\textit{#1}}}%
\newcommand{\hlopt}[1]{\textcolor[rgb]{0,0,0}{#1}}%
\newcommand{\hlstd}[1]{\textcolor[rgb]{0.345,0.345,0.345}{#1}}%
\newcommand{\hlkwa}[1]{\textcolor[rgb]{0.161,0.373,0.58}{\textbf{#1}}}%
\newcommand{\hlkwb}[1]{\textcolor[rgb]{0.69,0.353,0.396}{#1}}%
\newcommand{\hlkwc}[1]{\textcolor[rgb]{0.333,0.667,0.333}{#1}}%
\newcommand{\hlkwd}[1]{\textcolor[rgb]{0.737,0.353,0.396}{\textbf{#1}}}%
\let\hlipl\hlkwb

\usepackage{framed}
\makeatletter
\newenvironment{kframe}{%
 \def\at@end@of@kframe{}%
 \ifinner\ifhmode%
  \def\at@end@of@kframe{\end{minipage}}%
  \begin{minipage}{\columnwidth}%
 \fi\fi%
 \def\FrameCommand##1{\hskip\@totalleftmargin \hskip-\fboxsep
 \colorbox{shadecolor}{##1}\hskip-\fboxsep
     % There is no \\@totalrightmargin, so:
     \hskip-\linewidth \hskip-\@totalleftmargin \hskip\columnwidth}%
 \MakeFramed {\advance\hsize-\width
   \@totalleftmargin\z@ \linewidth\hsize
   \@setminipage}}%
 {\par\unskip\endMakeFramed%
 \at@end@of@kframe}
\makeatother

\definecolor{shadecolor}{rgb}{.97, .97, .97}
\definecolor{messagecolor}{rgb}{0, 0, 0}
\definecolor{warningcolor}{rgb}{1, 0, 1}
\definecolor{errorcolor}{rgb}{1, 0, 0}
\newenvironment{knitrout}{}{} % an empty environment to be redefined in TeX

\usepackage{alltt}
\usepackage[textsize=tiny,colorinlistoftodos]{todonotes}
%\usepackage{subfigure}
%\usepackage{subcaption}
%\usepackage{caption}
%\usepackage{tikz}
\usepackage{caption}
\usepackage{subcaption}
\usepackage{filecontents}
\usepackage{tikz-qtree}
\usepackage{pdflscape}
\usepackage{tikz}
\usepackage{multirow}
\usepackage{booktabs}
\usepackage{color, colortbl}

  \definecolor{Gray}{gray}{0.9}


\captionsetup[subfigure]{labelfont=bf,textfont=normalfont,singlelinecheck=off,justification=raggedright}
\input{preamble.tex}
%%\IfFileExists{upquote.sty}{\usepackage{upquote}}{}
\setlength{\mathindent}{0pt} %% NB MAYBEREMOVE BEFORE SUBMISSION

%\usetikzlibrary{arrows.meta}

%%%% TURN OFF PAGE NUMBER FOR INCLUSION IN THESIS
%\pagenumbering{gobble}
%%%%%%%
\IfFileExists{upquote.sty}{\usepackage{upquote}}{}
\begin{document}

%\begin{frontmatter}
%Lock, Stock and Two Smoking Barrels lock stock and two mediating factors
%The Lord of the Rings: The Two Towers
%Bravo Two Zero (1999)
%A tale of two mechanisms
%Eating strategy for breakfast?
%Push, skill and two mediating factors

%\title{One goal, two paths: How dynamic capabilities enable competitive advantage through behavioral and non-behavioral objects of change}
%\tnotetext[mytitlenote]{Fully documented templates are available in the elsarticle package on \href{http://www.ctan.org/tex-archive/macros/latex/contrib/elsarticle}{CTAN}.}

%% Group authors per affiliation:
%\author{Lars Hovdan Molden \fnref{myfootnote}}
%\address{Kongensgt 42, 7713 Steinkjer}
%\fntext[myfootnote]{Nord University Business School}

%% or include affiliations in footnotes:
%\author[mymainaddress,mysecondaryaddress]{Elsevier Inc}
%\ead[url]{www.elsevier.com}

%\author[mysecondaryaddress]{Global Customer Service\corref{mycorrespondingauthor}}
%\cortext[mycorrespondingauthor]{Corresponding author}
%\ead{support@elsevier.com}

%\address[mymainaddress]{1600 John F Kfennedy Boulevard, Philadelphia}
%\address[mysecondaryaddress]{360 Park Avenue South, New York}


% This paper investigates strategic cognition as mediating factor in the dynamic
% capabilities-competitive advantage relationship. By including strategic cognition into the
% analysis in addition to the conventional mediator (operating capabilities) we argue that a
% more holistic approach to understanding the workings of dynamic capabilities emerges. We
% introduce two expressions of how dynamic capabilities work, namely as a conductor (the
% conventional mechanism) and as a coach (the cognitive mechanism). Using repeated
% cross-sectional data from Norwegian technology firms our results indicate support for our
% holistic model. This has implications for understanding how dynamic capabilites work, as
% well as practical implications for how strategic change is implemented in firms.

{\large \bf One goal, two paths: How dynamic capabilities enable competitive advantage through behavioral and non-behavioral objects of change}\\
\newline 

% \begin{abstract}
{\bf Abstract:}\\
  The ultimate goal in the theory of dynamic capabilities (DCs) is competitive advantage.
  This paper analyzes to what extent DCs influence competitive advantage through two paths
  - one old, and one new - namely behavioral and non-behavioral objects of change.  In
  addition to emergent theorizing and real world examples, we use the metaphor of “DC as
  coach” and DC as conductor” to build a conceptual model and derive hypotheses.  Using
  longitudinal survey data with a 10 year time lag we find that DCs influence competitive
  advantage through both paths. Thus, the paper empirically validates emergent theorizing
  proposing non-behavioral objects of change as a new path between DC and competitive
  advantage. Moreover, the paper helps to broaden the conceptual basis for empirically
  investigating how DCs enable competitive advantage. This has implications for
  understanding how dynamic capabilities work, as well as practical implications for how
  strategic change is implemented in firms.

%\end{abstract}

\begin{keyword}
\texttt{Dynamic Capabilities, strategic cognition, operating capabilities, mechanisms,
  competitive advantage }
\end{keyword}

%\end{frontmatter}

%\linenumbers

%\doublespacing




\section*{Introduction}


%Frame flexibility: The role of cognitive and emotional framing in innovation adoption by incumbent firms rafaelli2019



%% Extant literature on dynamic capabilities is battling with a fundamental conception and
%% prediction of the theory, namely if, when and how dynamic capabilities can lead to competitive
%% advantage.
%% This paper is about how apsobritve capacity can come about as strategic posture
%https://arrow.dit.ie/cgi/viewcontent.cgi?article=1009&context=buschmancon

% paper about the DC for MNC in india an dimpact on CA
%https://journals.sagepub.com/doi/abs/10.1177/0971890717701781
The ultimate goal in the theory of dynamic capabilities ($DC$) is competitive advantage
\citep{Schilke2014,Helfat2007,Protogerou2012,Li2014,Efrat2018a,Mikalef2017,Davcik2016a}. Extant
theory argues that competitive advantage ($CA$) stem from resources and routines that are
well aligned to firms’ competitive environment
\citep{Helfat2007,Peteraf2013,Teece1997,Li2014,Jantunen2018,Makkonen2014}. Dynamic
capabilities are a source of longer-term competitive advantage in this perspective through
its capacity to purposefully change these lower-level routines and capabilities
\citep{Winter2003,Helfat2011,Schilke2018,Pezeshkan2016b,Lin2014a}, also referred to as
ordinary, substantive, and zero-level routines and capabilities
\citep{Winter2003,Danneels2008,Collis1994}. As a group, these lower level routines and
capabilities are called 'behavioral objects of change' \citep{Verona2011,Zollo2016}, and constitutes an
important path through which DCs can enable competitive advantage \citep{Verona2011}.

However, less emphasis has been placed on how other mechanisms can be work in shaping firm
competitive advantage. Indeed, this gap in the literature is explicitly emphasized by
\citep{Schilke2018} where they 'see an interesting opportunity for future work to add
greater richness to our understanding of the mechanism of resource-base change, given this
mechanisms central role in many foundational works (..) and the diverse ways in which
resource changes can potentially come the mechanism of resource-base change, given this
mechanisms central role in many foundational works (..) and the diverse ways in which
resource changes can potentially come about' (p. 419).

Indeed, the extant conceptualization of the $DC-CA$ link leaves out a considerable part of
a firm‘s 'resource base' that are non-behavioral, such as resources that are tied to how
the employees of the firm sees their mission, internalize information and effectuates
change. These traits of the organization is very much related to concepts of 'mindset'
\citep{Dweck2016}, ’employee cognition’ \citep{Lakoff1987,Gavetti2012} 'organizational
genes' \citep{Nelson1982} and 'sense-making' \citep{Weick1995} to name but a few. However,
in relations with dynamic capabilities, one promising avenue for exploring mechanisms,
beyond the conventional behavioral objects of change is to examine the role of
\emph{non-behavioral} objects of change in the theory of DC \citep{Verona2011}. These
non-behavioral objects of change contain psychological objects such as cognitive frames,
motivation and identity that are important for strategy formation and implementation
\citep{Vince2011,Gavetti2012} as they 'concern how decisions or actions are shaped,
subverted and transformed by emotions, and (..) determine the ”way we do things here”'
\cite[p. 338]{Vince2011}.

Moreover, these non-behavioral objects of
change are themselves far from static and are found to change over time as responses to
external change \citep{Zollo2016} thus making them a natural part of the mechanisms
underpinning how dynamic capabilities work \citep{Schilke2018,Verona2011}.  Still, far less
is know about the non-behavioral objects of change and how they work as a mechanism in the
dynamic capabilities theory.

We aim to contribute to bridging this gap by analyzing to what extent DCs influence
competitive advantage through two paths - one old, and one new - namely behavioral and
non-behavioral objects of change.  Specifically, we ask the following research question:

{\bf What is the role of non-behavioral objects of change as a mechanism through which
  dynamic capabilities can influence competitive advantage?}


To help answer this research question we build a conceptual model which illuminates the
relationships between $DC$, behavioral and non-behavioral objects of change and $CA$. The
model firmly integrates theorizing on organizational cognition \citep{Lakoff1987},
non-behavioral object of change \citep{Verona2011} and cognitive psychology
\citep{Tversky1983,Kahneman2011} into the $DC-CA$ nexus. Specifically, we focus on a
particular type of non-behavioral object of change, strategic cognition, and how it acts
as a mediator between dynamic capabilities and firm competitive advantage and hence has a
pivotal place in the literature. Reflecting this, we argue that DCs have two important
functions, namely 'DC as {\bf coach}' and 'DC as {\bf conductor}':

Dynamic capabilities take the function of a {\bf conductor} by orchestrating routines and
capabilities, and the role of a {\bf coach} in stimulating strategic cognition.  We discuss this
theoretical understanding of $DC$ and use the case of Microsoft’s transition into the cloud
computing business as an illuminating case. Importantly, this case demonstrates how the
roles of conductor and coach coexist and in generating strategic change and competitive
advantage.

The conceptual model is used to derive a set of hypotheses which is tested against
longitudinal survey data with a 10 year time lag. We find that $DC$ influence competitive
advantage through both behavioral and non-behavioral objects of change. Thus, the paper
empirically validates emergent theorizing proposing non-behavioral objects of change as a
new path between $DC$ and competitive advantage. Moreover, the paper helps to broaden the
conceptual basis for empirically investigating how $DC$ enable competitive advantage. This
has implications for understanding how dynamic capabilities work, as well as practical
implications for how strategic change is implemented in firms. Thus, the paper clarifies
the theoretical role of non-behavioral objects of change as well as a possible
operationalization through the concept of \emph{strategic cognition}. Moreover, we
conceptualize two different expressions of dynamic capabilities (i.e. conductor and coach)
and theorize on how they work in the $DC-CA$ nexus. Finally, we develop and test an empirical model
suited for analyzing mediating constructs on the dynamic capabilities-competitive
advantage nexus. The time-lag of our data in conjunction with advances in empirical
estimations of multiple simultaneous mediators makes for an interesting model to push our
understanding of important mechanisms in the theory.

The rest of this paper is organized as follows: First, we present a brief discussion on
the key definitions applied in this paper. We then move on to develop hypothesis of the
focal relationships. Third, we present the data and empirical models used to test our
hypothesis. A discussion follows the results from our model, before we conclude by
offering some limitations to our study as well as future directions for study.

\section*{Theoretical background and hypotheses}

Competitive advantage is considered the ’holy grail’ of strategic management
\citep{Schilke2018} and is indeed the focal point of important antecedent theories of
dynamic capabilities \citep{Arndt2018,Pezeshkan2016b}. Consequently, contributions to push
the theory forward should enable a better understanding of how dynamic capabilities work
to influence firms’ competitive advantage.

The collective term for assets available for firms in their pursuit of their objectives
are \emph{resources} \citep{Helfat2011,Helfat2007}. This term entails tangible assets such
as machines, capital, and labor, but also intangible assets such as procedures, practices,
and intellectual property \citep{Pisano2017}. The process of using these resources in the
pursuit of business objectives such as competitive advantage is a firm's
\emph{capabilities} which can be a set of routines and skills applied
\citep{Helfat2007,Helfat2011}. Resources and capabilities are both tied organizational
behavior and how the organization act in their operation. Hence, they are coined
\emph{behavioral objects of change} \citep{Verona2011}: 'A pattern of action, a process,
an operating routine, or any form of group activity characterized by some level of
stability and predictability' (ibid. p.538). Such behavioral objects such as capabilities
are the conventional understanding of how dynamic capabilities work; they 'purposefully
create, extend and modify the resource base' \cite[p. 4]{Helfat2007}. In the terminology
proposed in this paper this process takes the shape as a {\bf conductor} envisioning a
conductor orchestrating firm capabilities. Extant literature on dynamic capabilities are
increasingly focusing on how firm resources are changed by dynamic capabilities as a
response to changes in the environment \citep{Schilke2018,Eriksson2014}, thus placing firm
capabilities as a mediator between dynamic capabilities and competetive advantage
\citep{Helfat2007}. This notion of 'orchestration' of firm resources, assets and
capabilities has indeed been emphasized in the extant literature
\citep{Sirmon2011,Helfat2015}. Thus the notion of a {\bf conductor} captures a directional
role played by $DC$ in terms of its functioning to 'orchestrate' \emph{behavioral}
objects of changes such as \emph{operational capabilities}.

However, as \cite{Verona2011} points out: 'This approach misses the fundamental aspect
that organizational capabilities specific to managing change can hardly be reduced to the
management of behavioral change' (ibid p. 538). This opens for \emph{non-behavioral
  objects of change} which are defined as 'behavioral antecedents, such as cognition
(e.g. cognitive frames) and motivation' (ibid p. 539). This concept entails the 'how
decisions or actions are shaped, subverted, and/or transformed by emotions; and it
concerns how emotions become embedded in cultural and political practices that determine
the "way we do things here"' (Vince and Gabriel 2011 . ch 15). One particular dimension of
these non-behavioral objects of change particularly conducive to the context of
generating strategic change is the \emph{strategic cognition} of the
organization. Strategic cognition can be defined as conceptual structures in the minds of
the individuals that encapsulate a shared understanding of the reality the
individuals face with respect to the strategic goal of the organization \footnote{This
  definition is based on the definition of \cite{Lakoff1987} definition of the more generic
  'cognitive representation' and is adapted to fit the pursuit of strategic goals such as
  competetive advantage which is the aim of this paper}. This strategic
cognition is the result of a mental process that underlie internal
audiences adoption of new strategic representation and identity code to move toward a
new or existing goal \citep{Gavetti2012}. Although strategic cognition may lead to
behavioral outcome, the nature of the concept itself is tied to cognition rather than
actual latent action \citep{Zollo2016}. Thus, dynamic capabilities are not sufficiently
creating strategic change through its role as {\bf conductor}. Additionally they need to
exercise a role to adapt the strategic cognition of the organization to create
a shared understanding of the sense and necessity of the strategic goals of the
orientation. We coin {\bf coach} to capture the enabling role dynamic
capabilities plays through \emph{strategic cognition} as
juxtaposed to the more directional role played as {\bf conductor} of \emph{operational
  capabilities}. According to the Cambridge dictionary the noun 'coach' means 'someone
whose job is to teach people to improve' and 'someone whose job is to train and
organize'. Albeit often used in relation to sports it is also increasingly used for any
process of 'learning and development intervention that uses a collaborative, reflective, goal-focused
relationship to achieve professional outcomes' \cite[p. 253]{Jones2016}. We use the word
as an analogous idiom for an organizational process focusing on the subject (e.g. the firm or an
individual) to utilize own capacities as well as activating latent motivations. Thus it is
juxtaposed to a {\bf conductor} that creates results through directions and
orchestration.

\subsection*{Conducting and coaching at Microsoft}\label{sec:case}


When Satya Nadella took over as CEO of Microsoft in March 2014, he took over a company in
big trouble. The share price was just north of 30 USD and the company had just launched a
7.2 billion USD acquisition of the failing Nokia company. The story about how Nadella
turned Microsoft around to a valuation of 105 USD pr share and with successful mergers
along the way is a story of excellent 'capabilities orchestration' as a response to exercising
dynamic capabilites, specifically through sensing, seizing and transforming resources to
adapt to changing market conditions \citep{Teece2007}. Nadella and his firm did a
considerable reorientation towards the cloud and reconfigured firm capabilities to fit
this objective - almost a school example of exercising dynamic capabilities.

Moreover, and to the point of this paper, through a series of talks, interviews and his
own book Nadella put considerable emphasis on how Microsoft managed to change their
'mindset', that is their 'cognitive frames' \citep{Nadella2017}: 'I focused on what would be
our grandest endeavor, the biggest hurdle - transforming the Microsoft culture'
(p. 89). Drawing on the concept of a 'growth mindset' \citep{Dweck2016} Nadella sees
culture as a 'complex system of individual mindsets' and that it encompasses how 'an
organization thinks and act' (ibid p.90).

Satya Nadella is very much on the same page as recent contributions on how cognitive
frames of an organization (i.e. its non-behavioral object of action) is a pivotal part in
how firm resources shape outcomes such as performance and competitive advantage
\citep{Barney1991a,Barney2001}, and how it consequently plays a role in how dynamic
capabilities work along these lines \cite{Verona2011}. Microsoft under Nadella did
transform the company by changing the behavioral objects of action (i.e. capabilities and
routines), but also through changing the non-behavioral objects of action (i.e. the
strategic cognition). In other words, the story of Microsoft`s transformation suggest that
dynamic capabilities worked through mediating factors to influence competetive advantage,
but that these mediating factors exist as two distinctly different concepts working in
tandem. Figure \ref{fig:fig1} presents a conceptual depiction of the paths running through
these two concepts of \emph{capabilities} and \emph{strategic cognition} objects of action
\citep{Zollo2016}. We use a familiar term for the conventional understanding of dynamic
capabilities mechanism as 'orchestration' as a collective term to describe the
behavioral objects of action \citep{Verona2011}(ref). In this process dynamic capabilities play the role of
a {\bf conductor} orchestrating capabilities of the firm, an understanding that is in line
with a range of the most common definitions of dynamic capabilities \citep{Schilke2018}.

The non-behavioral objects of action has no similar agreed upon analogy, so for the
purpose of our analysis we envision this mechanism to work more as a cognitive process
enabling the organization to tear away from their current understanding and ways in which
things are done \citep{Vince2011}. Whereas the analogy of a {\bf conductor} suggests a
function that composes an outcome based on the available capabilities, the notion of a
{\bf coach} refers to the function of enabling others to achieve their objectives. The
process is all about getting employees on board, and by changing an organization`s
thinking from the automated and well-known, to the more cognitively aware. We will now
turn to the two functions of dynamic capabilities in turn.


\subsection*{Capabilities conductor}

Satya Nadella set out to transform Microsoft by investing heavily in artificial
intelligence and cloud computing capabilities. This entailed hiring new people, upgrading
the skills and abilities of the current staff, and acquire infrastructure and
hardware. Additionally, Microsoft was able to acquire and integrate a number of companies
to support the strategic direction the company was on. Companies such as GitHub, LinkedIn
and Citrus data became important platforms for developing new capabilities within the
company \cite{Nadella2017}.

This role of {\bf conductor} is the conventional way of understanding dynamic capabilities. It
stems from one of the most common definitions \citep{Helfat2007} as well as the analogy of
the process as 'resource orchestration' \citep{Sirmon2011}. It is also found in the
empirical literature handling various functional forms of dynamic capabilities such as
acquisition targeting \citep{Bingham2015}, alliancing \citep{Schilke2014}, new product
development \citep{Danneels2008}, innovation capabilities \citep{Breznik2014}, marketing
capabilities \citep{Brune2009} and alliance capabilities \citep{Kale2002,kale2007}.

The bulk of prior contributions study firm capabilities and resources as the mechanism
through which dynamic capabilities influence an outcome
\citep{Pezeshkan2016b,Fainshmidt2016a,Protogerou2012,Schilke2018}. Particularly, much of
the extant literature concerns measures of accounting profitability and competitive
advantage as the final outcome of dynamic capabilities
\cite{shamsie2009,Schilke2014,Schilke2014a,Teece2016}. Thus, from these prior
contributions we derive a fundamental hypotheses of the relationship between dynamic
capabilities and competitive advantage:

\emph{H1: Firm capabilities mediates the relationship between dynamic capabilities and
  competetive advantage}

However, this mechanism is solely related to behavioral objects of action while previous
insights from the literature on organizational capabilities hold that this is only parts
of the story \cite{Tripsas2000,Gavetti2012}. Or as highlighted by \cite{Verona2011}: Most
'conceptualizations of DCs have in common the often implicit assumption that the object
upon which DCs produce their effects is fundamentally of a behavioral nature: that is a
pattern of action, a process, an operating routine, or any form of group activity
characterized by some level of stability and predictability. This approach misses the
fundamental aspect that organizational capabilities specific to managing change can
hardly be reduced to the management of \emph{behavioral} change' (p. 538). Thus, they
argue, a more holistic approach including cognition is needed \citep{Zollo2016}. This claim is also
supported in the lessons from Microsoft's transition.

% https://hbswk.hbs.edu/item/the-transformation-of-microsoft


\subsection*{Cognitive coach}

On February 4th 2014 an email appeared in the inbox of all Microsoft employees. It was
from Nadella who had just begun his tenure as CEO of the company. In this email he lays
out the framework for transforming Microsoft around changing the way the company jointly
acted. He wanted to change the culture of the company from one of inviting new ideas and
fostering creativity. This called for changes in the \emph{cognition} of the individual
employers as well as their capabilities and skills. Nadella made this abundantly clear in
this email even paraphrasing Oscar Wilde in that 'we need to believe in the impossible and
remove the improbable'. Nadella's tool for achieving this change of cognition was to
become the notion of the growth mindset \citep{Dweck2016}. This transformation of the
cognition of the organization to better fit the strategy he was laying out, is a proper
example of instilling a \emph{strategic cognition} in the organization.

vJuxtaposed to the experiences from Microsoft, the case of Polaroid brings another set of
experiences to our discussion on non-behavioral objects of action (see \citep{Tripsas2000}
for a detailed account). Cognitive inertia on the part of the organization played a
pivotal part in the failing of the company`s digital imaging effort \citep{Verona2011}. In
other words, Polaroid did not exercise dynamic capabilities to change the cognition of the
organization. Similar patterns are found in Kodak and Anderson Consulting
\citep{Kaplan2005}, and Smith Corona \citep{Danneels2010}. These examples highlight the
absence of 'dynamic capabilities specific to the adaptation of cognitive frames, first,
and consequently of operating processes' \citep[p. 541]{Verona2011}. This is also
highlighted at the core of the dynamic capabilities literature: 'Dynamic capabilities are
about doing the right things, at the right time, based on new product (and process)
development, unique managerial orchestration processes, a strong and {\bf change-oriented
  organizational culture}' (our emphasis) \cite[p]{Teece2014}.

When dynamic capabilities are executed it is as a response to changes in the environment
in order to stay competitive. This processes relates to creating strategic change
\citep{Helfat2007} and entails changing strategic direction and objectives of the
organization. This strategic change requires, in addition to modifying, extending and
creating firm capabilities, also some kind of alignment with the organization as a whole -
what is often referred to as 'how we do things here' \citep{Nelson1982}, and how an
organization think and act \citep{Nadella2017}. This alignment is possible only if the
employees share a common cognitive frame related to the the strategic direction of the
company enabling them to make sense of the changes ahead. This is what we have coined
\emph{strategic cognition} and has its expression in Microsoft's development of a growth
mindset. Strategic cognition is needed to get employees on board as well as capture the
efficiency of non-directed change. Getting on board refers to the general willingness of
the organization to accept and value the importance of some decision or system 'without
being prompted or required by formal governance
mechanisms'\citep[p. 812]{Simsek2009}. Moreover, When individuals of a firm buys into
changes they will enhance the full organization`s ability to not only act upon changes
\citep{Marchand2004}, but also in 'timely responsiveness' \citep{Kohli1993}.

Seminal contributions in cognitive psychology describes this state of actively engaging in
cognitive processes 'system 2 thinking' and define it as allocating 'attention to the
effortful mental activities that demand it, including complex computations'
\cite[p. 21]{Kahneman2011}. This is juxtaposed to the more automated 'system 1' which
'operates automatically and quickly, with little or no effort and no sense of voluntary
control' (ibid). The process is all about changing an organization`s thinking from the
automated and well-known, to the more cognitively aware. The extent to which individuals
in an organization is enabled to allocate attention to effortful mental activities related
to understanding needs for strategic change instead of relying on automatic reasoning
\citep{Tversky1983} or rules of thumb \citep{Kahneman1979}, can be a forceful remedy
aganist organizational inertia \citep{Adriaenssen2016}.

Consequently, strategic cognition not only contributes to getting people on board, it also
maintain a more efficient change process where the individuals take more responsibilities
for the change itself and leverages their individual knowledge in the endeavor. For a firm
to create real strategic change as enabled by their dynamic capabilities and hence stay
competitive, is will need to act as a {\bf coach} to instill a strategic cognition in the
organization.

\emph{H2: Strategic cognition mediates the relationship between dynamic capabilities and
  competitive advantage}

\subsection*{Coaching conductor}

Although Satya Nadella and Microsoft put strong emphasis on the cognitive side of
strategic change (the role as {\bf coach}), acquisition and development of capabilities
were also very much on their mind (the role as {\bf conductor}). It was not about letting
'culture eat strategy for breakfast', but rather serving culture and strategy as
complementary side-dishes. Nadella emphasized 'getting the right team in place', 'build
new and surprising partnerships', 'be ready to catch next wave of innovation and platform
shift' as things Microsoft needed to get right. These meant investing in new people as
well as upgrading the capabilities of the existing staff \cite{Nadella2017}. Meanwhile, he
also stressed that they needed to 'drive cultural change from top to bottom' and 'reframe
our opportunity for a mobile- and cloud-first world'. (ibid p 95-96). In Microsoft, it
clearly seems, it was exactly by being both a {\bf coach} and a {\bf conductor} at the
same time that results would come. In other words, the giant software company strongly
emphasized that the two mechanisms were complementary.

When dynamic capabilities are executed in a firm being able to both adapt the capabilities
and the cognitive frames simultaneously makes sense. A solid strategic cognition in tandem
with relevant and strong capabilities would reinforce each other. This has also been
supported in previous research where firm success has been shown to relate to the
interconnection between resources and capabilities, and culture and attitude
\citep{Verona2003}, and high levels of social capital \citep{Blyler2003}. Moreover, the
lack of one (e.g. strategic cognition) can lead to unfavorable outcomes even though
capabilites and resources are aligned for the needed strategic change
\citep{Tripsas2000}. This leads us to our final hypothesis:

\emph{H3: Strategic cognition and operating capabilities are complementary and thus
  mutually reinforcing their individual effect on competitive advantage}

\begin{figure}
  \centering
  \captionsetup{width=0.5\linewidth}
  \begin{tikzpicture}
   \centering
   \tikzstyle{mynode}=[circle,draw, ultra thin, draw=black, fill=white, minimum
   size=17mm,inner sep=0pt, align=center]

   \tikzstyle{mytext}=[fill=none, minimum
   size=17mm,inner sep=0pt, align=center]
   
      
        
   \node[mynode] (dc1){$DC$};
   
   \node[mynode,right of=dc1, yshift= 2cm, xshift=2.5cm] (cap){$CAP$};
   \node[mynode,right of=dc1, yshift=-2cm,xshift=2.5cm] (cog){$COG$};
   \node[mynode,right of=dc1, xshift=6cm] (ca){$CA$};
   \node[mytext,above of=dc1,rotate=36, xshift=1.6cm,yshift=0cm] (ro){\bf Conductor};
   \node[mytext,above of=dc1,rotate=-36, xshift=2.8cm,yshift=-1.5cm] (cp){\bf Coach};
    
   \draw[-latex] (dc1.north east) -- node[text width=0.5cm,font=\footnotesize, above=2pt,align=left,
   fill=none] {} (cap.west);

    \draw[-latex] (dc1.south east) -- node[text width=0.5cm,font=\footnotesize, above=2pt,align=left,
     fill=none] {} (cog.west);

    \draw[-latex] (cap.east) -- node[text width=0.5cm,font=\footnotesize, above=2pt,align=left,
      fill=none] {} (ca.north west);


    \draw[-latex] (cog.east) -- node[text width=0.5cm,font=\footnotesize, above=2pt,align=left,
    fill=none] {} (ca.south west);

        \draw[-latex] (dc1.east) -- node[text width=0.5cm,font=\footnotesize, above=2pt,align=left,
      fill=none] {} (ca.west);



      

   %\draw[-latex] (dc2.east) -- node[text width=0.5cm,font=\footnotesize, above=2pt, align=left,
   %fill=none] {} (cap.west);

   %\draw[-latex] (dc1.south east) -- node[text width=0.5cm,font=\footnotesize,  above=2pt,
   %align=left, fill=white,sloped, near end] {$0.496^{***}$} (ma.north west);

   %\draw[-latex] (dc2.north east) -- node[text width=0.5cm,font=\footnotesize, above=2pt,align=left,
   %fill=white,sloped, near end]{$0.343^{***}$} (or.south west);

    %\draw[-latex] (or.east) -- node[text width=0.5cm,font=\footnotesize, align=left,above=2pt,
   %fill=white,sloped]{$0.185^{***}$} (ca.north west);

   %\draw[-latex] (ma.east) -- node[text width=0.5cm,font=\footnotesize, below=2pt,
   %fill=white,sloped]{$0.123^{**}$} (ca.south west);

   %\draw[-latex] (dc1.north)[bend left] to node[text width=0.5cm,font=\footnotesize, above=2pt,align=center,

   %fill=white]{$0.299^{***}$}(ca.north);

   %\draw[-latex] (dc2.south)[bend right] to node[text width=0.5cm,font=\footnotesize,below=2pt, align=left,
   %fill=white]{$-0.066^{}$} (ca.south);
    
 
\end{tikzpicture}

  \captionof{figure}{The resource and cognitive mechanisms of dynamic capability}
  \label{fig:fig1}
\end{figure}


\section*{Data and methods}

To test the relationships between dynamic capabilities and competitive advantage, as well
as the mediating effects on operational capabilities and strategic cognition, we need a
proper metric of all. Particularly, it is useful to consider changes over time to capture
the dynamic in the relationships and make for a more plausible causal argument to be put
forth. The lack of longitudinal studies to investigate dynamics over time and control for
path dependency of organizational routines has been noted as permanent limitation for
moving the dynamic capabilities literature forward empirically and theoretically
\citep{Schilke2018}.


We use a unique data set where Norwegian technology firms were administered a survey at
two points in time. ($T_0 = 2005$ and $T_2 = 2014$). The data combine the secondary data
from an earlier survey with registery data and a follow-up survey conducted by the
authors. The baseline survey was captured in 2005 when a web based survey was sent to 1721
technology firms in Norway intending to capture the concept of dynamic capabilities
\citep{Alsos2007}. Approximatly 70\% of the firms returned filled-in questionaires mouting
to a total sample size of 1199 firms. Details of the data gathering process and the
validation is available in \cite{Alsos2007}, but a brief explanation is waranted. The
authors used a mixed methods approach combining an extensive literature review with
qualitative interviews and testing in the field. Face validity of the items was further
examined by testing the items on experts. The authors of the present paper did take part
in this data initial collection. However, we have been given access to the raw data
collected in 2005. The second author of this paper followed up with a similar survey in
the spring/summer of 2014, 9 years after the first round. All the same firms received a
web-based questionnaire containing the same measures of the key constructs in this
paper. Following htis second round we were left with a sample size of 283 representing the
number of firms returning a filled-in second questionnaire.

The population was all businesses registered to a scheme for
tax deduction of R\&D costs (called SkatteFUNN). As all enterprises which are eligible for
taxation could register their R\&D activities to receive a tax refund, the registered
enterprises include close to all enterprises which are involved in such activities at the
time of our study. The context of R\&D active firms is particularly relevant for studying
dynamic capabilities because R\&D activity is a signal of responsiveness to change and
that the firms are located in industries with a certain level of dynamism.

All the firms in the sample were identifiable by means of the official firm
identifier. This enabled us to attach financial data from the annual accounts of the firms
despite none of the firms in the sample being publicly traded companies. We obtained the
financial accounts from the National Firm Registry (BRREG) for all years between 2007 and
2014 (prior to 2007 was not accessible through our database). By including data from a
second source, we aim to remedy to known short comings of similar
contributions. See the section of control variables for short
discussion.

%% First, we
%% remedy some of the problems possibly arising from common method bias (ref) and
%% BLABLABLA. Capturing the same aspect of competitive advantage from several sources
%% (i.e. both from our survey as well as through the reported profit margin from the
%% Norwegian Firm Registry) makes the findings more reliable in terms of not relying solely
%% on the perception of the respondents in the survey, but controlling for more objective
%% metrics with similar characteristics. Particularly, controlling for factors such as firm
%% profitability and balance sheet size, MORE HERE. Second, a common critique of the dynamic
%% capabilities - competetive advantage relationship is that they both have a common
%% confounded in simply coming from being a 'good firm'. In other words, critics argue that
%% the reason we find strong relationships between these two are that they are the same and
%% we are simply regressing $X$ on $X$. By including more objective measures of what
%% constitutes a 'good firm' (e.g. profitability) we partly get at the common
%% critique. Combined with a time lag both between $X$ (dynamic capabilities) and $Y$
%% (competitive advantage), as well as $Y$ observed over time, we at least partly control for
%% common factors. Moreover, we add robustness checks for endogeneity as a n extra layer of
%% certainty (see the result section).

\subsection*{Variable construction}

The focal constructs of this paper are dynamic capabilities, operating capabilities,
strategic cognition as measured by proactiveness, and competitive advantage. These are all
operationalized using the survey data described above.  To capture \emph{operating
  capabilites} we look at the definition of these by \citep{Winter2003} as those
capabilities that enables the firm to 'make a living' in the short term, i.e. contributing
to the technical fitness of the firm \citep{Helfat2007}. Thus we chose to survey a firm's
ability to handle core functions of daily operations such as operational routines,
operational management, customer service and marketing. To capture strategic cognition we
apply the definition derived above. Of course such a concept can have many
actual expressions, but in the context of studyin competititve advantage the
organization's attitude towards competition seems like a plausible starting point. In
other word a shared cognitive frame of an overall competitive orientation is useful. Thus,
we adopt the idea of measuring a strategic posture of the firm \citep{Covin1989} and
particularly the notion of \emph{proactiveness} \citep{covin1991,Wales2016}.


Table \ref{tab:item} in the appendix describes the items used to operationalize the focal
constructs and their internal validity as measured by Dillon-Goldstein Rho which is the
preferred metric for this purpose \citep{Hair2019}. The results show that our key
constructs can be represented by the items in the survey, with a minor caveat. The measure
of strategic cognition as proactiveness loads only partly on one item (item
3). Statsistically, a loading of $0.53$ is on the weaker side of
traditional 'rules of thumb' \citep{Hair2010}, but from a theoretical stand point the
measure makes sense and is established in the literature \cite{covin2012}.


There are no established measurement models of dynamic capabilities
\citep{McKelvie2009,Schilke2018}.Therefore this study builds on a mixture of qualitative
case study methodology, literature review and statistical techniques to develop and refine
measures of DCs. First, exploratory qualitative interviews, using a semi-structured
interview guide, were conducted with management representatives from 10 R\&D/innovative
firms. The aim of the interviews was to get an overview of each firm's innovation and
development processes, in particular the processes related to its dynamic capabilities and
the management of its resources. Themes raised in the interviews were about network,
co-operation with external R\&D-institutions, learning in the firm, adaptation and changes
in the firm. We interviewed SMEs and larger firms, and the industries varied from
high-tech and ICT to publishing.  Based on the interviews and an extensive literature
review, statements identified as descriptions of dynamic capabilities and resources were
developed and included in a questionnaire.

Second, the informants from the 10 firms were subsequently asked to take part in a pretest
of the questionnaire, including the preliminary items developed to measure dynamic
capabilities and resources, by responding to the questionnaire and giving comments on the
individual questions. This was followed up by a telephone call to the interviewees where
they were asked to report their views on the various questions/items.

Third, the face validity of the items was further examined by pre-testing the measurements
among experts. Researchers with knowledge of business strategy within firms were asked to
evaluate the questionnaire. Based on the results from the pilot study, the items were
adjusted and refined.

Although we recognize at the outset that the concept of dynamic capabilities and their
underlying resource components are very challenging to research in a systematic and
econometric fashion \citep{McKelvie2009}, we follow the argument in the literature that
more empirical work is necessary to test and refine the dynamic capabilities concept and
how it is related to the evolutionary economic theory \citep{Arend2009,McKelvie2009}.  It
is in this spirit that the research reported in this paper has been undertaken.
% \todo{Rosenblom, 2000; Verona and Ravasi, 2003;}

Respondents were asked to indicate to what extent each statement fitted a description of
their business. We adopted one-sided seven point Likert scale where: 1 = strongly disagree
and 7 = strongly agree. We built on prior studies where items measuring operating
capabilities have been measured relative to competitors \citep{McKelvie2009}.

The nine year lag between our points of measurement has its distinct advantages with
respect to capturing long term trajectory of the firms. For example, the operating
capabilities at time $t=1$ is very likely a function of operating capabilities at time
$t=0$ as well as other factors such as dynamic capabilities. Indeed, extant literature
argues for this path dependency of resources and routines
\citep{Helfat2007,Winter2003}. These capabilities are said to be serial-correlated and
path dependent. However, as dynamic capabilities and operating capabilities are distinctly
different concepts, this kind of path dependency is also not equally clear in this
regard. We could have firms with very high levels of dynamic capabilities at $t=0$ and
equally a poorer level at time $t=1$. Because of the long time lag a lot of factors could
have influenced this decline, and indeed a rather high number of over firms exhibit this
characteristic of declining dynamic capabilities over time. As we regard dynamic
capabilities as 'the first derivative' \citep{Winter2003} (i.e. the rate of change of the
operating capabilites) a large initial value (at time $t=0$) could over estimate the
expected change of the underlying capability at time $t=1$. Moreover, based on the scale
applied (Likert from 1 through 7) a high initial value leaves nowhere to move but
downwards. This is not to say that those firms are likely to develop poorer operating
routines. Nor is to say that those with low dynamic capabilities at time $t=0$ are
rendered useless in developing capabilities going forward towards $t=1$. Rather it
suggests that a long time span may include unnecessary noise in the model. Thus, we
estimated an average of the dynamic capabilities of the firm between time $t=0$ and $t=1$
as a proxy for a more representative measure of the rate of change. Hence, our construct
of $DC_0$ is actually a measure between times $t=0$ and $t=1$. As a robustness check we
repeated the full analysis using 2005 measure as our $DC_0$. This did not severely alter
the conclusions we draw in this paper, but had some minor effects on the significance of
the path between $DC_0$ and $COG_1$.


\subsection*{Control variables}

We add a set of control variables expected to explain parts of the variation in our focal
constructs. From the survey we add firm size and firm age. Size is the reported number of
employees in the firm at $T_1$, and age is the number of years since establishment.

To partly remedy the problem of common method bias, we add controls of financial data from
the firm´s profit and loss statements, and balance sheet as robustness
checks. Specifically, we add the natural logarithm of the assets to capture the capital
intensity of the firm.

One core critique of the dynamic capabilities literature is that it simply is a response
to the firm being a good firm \citep{Arend2009}. This tautology makes it hard to
distinguish the direction of the effects the performance implications of dynamic
capabilities. Simply put both firm performance and dynamic capabilities can be due to some
unobserved factor of the firm simply being a 'good firm'. In such cases we would simply
regress one on the other and get positive results without the exact mechanism being
pointed out. To partly control for this we add a proxy for the quality of the firm as a
control variable. In this paper we use the profit margin of the firm as this type of
control. It is not perfect as profit margins can be due to many factors outside our model,
but it serves to control for parts of the possible cofounding factors of firm quality.

All the constructs and the control variables are presented with descriptive statistics and
pairwise correlations. The main estimation was tested for multicolinearity using the
variance inflation test (VIF) with no severe multicolinearity detected. This is also
suggested by the correlation matrix.

\begin{table}[t]
  \caption{Descriptive statistics and correlation between key constructs and control
  variables.}
\label{tab:cor}

\singlespacing
\tiny
\setlength{\tabcolsep}{3pt}
% latex table generated in R 3.6.1 by xtable 1.8-4 package
% Tue Jan 28 12:23:38 2020
\begingroup\scriptsize
\begin{tabular}{lp{0.290in}p{0.290in}p{0.290in}p{0.290in}p{0.290in}p{0.290in}p{0.290in}p{0.290in}p{0.290in}p{0.290in}p{0.290in}p{0.290in}p{0.290in}}
  \hline
 & Mean & SD & 1 & 2 & 3 & 4 & 5 & 6 & 7 & 8 & 9 & 10 & 11 \\ 
  \hline
1) $CA_1$ & 3.89 & 1.39 & 1 &  &  &  &  &  &  &  &  &  &  \\ 
  2) $CA_0$ & 3.88 & 1.47 & 0.38 & 1 &  &  &  &  &  &  &  &  &  \\ 
  3) $DC_0$ & 4.66 & 0.84 & 0.3 & 0.06 & 1 &  &  &  &  &  &  &  &  \\ 
  4) $CAP_1$ & 4.17 & 0.99 & 0.54 & 0.18 & 0.36 & 1 &  &  &  &  &  &  &  \\ 
  5) $CAP_0$ & 4.23 & 1.17 & 0.32 & 0.41 & 0.21 & 0.36 & 1 &  &  &  &  &  &  \\ 
  6) $COG_1$ & 4.63 & 1.15 & 0.45 & 0.06 & 0.39 & 0.46 & 0.25 & 1 &  &  &  &  &  \\ 
  7) $COG_0$ & 4.71 & 1.26 & 0.19 & 0.27 & 0.24 & 0.15 & 0.37 & 0.34 & 1 &  &  &  &  \\ 
  8) Size & 4.33 & 1.90 & 0.36 & 0.29 & 0.06 & 0.31 & 0.29 & 0.11 & 0.13 & 1 &  &  &  \\ 
  9) ln(Assets) & 10.01 & 2.03 & 0.35 & 0.33 & 0.06 & 0.32 & 0.26 & 0.08 & 0.11 & 0.87 & 1 &  &  \\ 
  11) Margin & -0.07 & 0.66 & 0.18 & 0.24 & 0 & 0.09 & 0.13 & -0.06 & -0.09 & 0.07 & 0.02 & 1 &  \\ 
  12) Age & 25.12 & 14.24 & 0.17 & 0.18 & -0.05 & 0.19 & 0.17 & -0.02 & -0.05 & 0.52 & 0.51 & 0.04 & 1 \\ 
   \hline
\end{tabular}
\endgroup



\end{table}

\subsection*{Estimation and empirical model}

Our empirical model is a mediation analysis where operational capabilities and
proactiveness acts as mediators of the dynamic capabilities-competitive advantage
relationship. Our data is for the larger part survey based with some controls gathered
from registry data (particularly the Norwegian Company Registry). The total sample size of
the paper is $N=262$ which makes it very suited for a latent variable analysis using
Partial Least Squares - Structural Equation Modeling (PLS-SEM)
\citep{Hair2010}. Particularly, PLS-SEM is useful when studying theoretical extensions of
established theories, when the sample size is small, when the structural model includes
many constructs and indicators, and when distribution issues are a concern
\citep{Hair2019}. Indeed, the use of PLS-SEM is increasingly taking hold in strategic
management research \citep{Hair2012}.

As we work with latent constructs we needed to handle missing values in the measurement
model. Each latent variable consists of several observable indicator variables and the
latent variable will drop an observation if one of the indicators are missing. Thus, to
reduce the number of missing observations we ran a two-step validation process when
constructing the latent variables. First, we established a theoretical structure based on
theory and tested the constructs accordingly by evaluating the internal validity using
Dillon-Goldstein Rho. In this first step we used only observations where all indicators
were complete (i.e. no missing). Then we imputed the mean value of the other indicators
when one indicator had a missing value. In the second step we reran the model and compared
the reliability measures. None of the constructs changed materially, but the total sample
size went from $N=262$ to $N=240$ because some of the constructs lacked more than one
item, rendering the imputation challenging. In these cases we dropped the observation. As
a robustness check we reran all the models using a Multivariate Imputation By Chained
Equations (MICE) procedure which makes use of the whole sample to impute missing values
where more than one underlying item is missing. This procedure did not change the results
materially.

The complete data used in the measurement model then comprised of $N=240$
observations. When adding the registry data from the Norwegian Company Registry, the
sample sized declined somewhat to $N=206$. We ran all the models with and without the
registry data without the focal relationship changing materially.

To test our hypotheses we then built a path-model from the latent variables we
constructed. This resulted in four models. The results from the structural model is
reported in figure \ref{fig:med4} and table \ref{tab:res}. To estimate the standard errors
we ran simulations using bootstrapping and reported the simulation number, standard errors
and corresponding p-values. To evaluate the models we compared $R^2$ and $Q^2$ statistics
for model comparison. All these are reported in table \ref{tab:res}.

%% Finally, we tested the model for nonlinear effects,
%% endogeneity and unbserved heterogeneity using Regression Equation Specification Error Test
%% (RESET), Gaussian Copula approach and Information Criteria test respectively. These
%% results are available in the appendix.

The estimated models are based on the following specification:
\begin{align}
  CA_{i,1}  = & \alpha_0 + \alpha_1 CA_{i,0} + \omega_1 \alpha_2 CAP_{i,1} + \\
           & \omega_2 \alpha_3 COG_{i,1} + \alpha_{4} DC_{i,0} + \mathbf{\Gamma}_m\mathbf{C}^m_{i,1} + \epsilon_i
\end{align}

\begin{align}
  CAP_{i,1} & = \beta_0 + \beta_1 CAP_{i,0} + \beta_2 DC_{i,0} + v_i\\
   COG_{i,1} & = \eta_0 + \eta_1 COG_{i,0} + \eta_2 DC_{i,0} +  u_i
\end{align}

$CA$,$CAP$,$COG$ and $DC$ indicates the consctruct for competetive advantage, operational
capabilities, strategic cognition (proactiveness) and dynamic capabilities
respectively. Subscript $i$ refers to the firm and the second subscript indicates the time
$t=0$ or $t=1$. $\mathbf{\Gamma}_m$ is an $m \times 1$ vector of coefficients $\gamma_1$
to $\gamma_m$ for the control variables included in the model captured in
$\mathbf{C}^m_{i,1}$ which is a $i \times m$ matrix of controls. This generic model
structure corresponds to four models by changing the parameters $\omega_1$ and $\omega_2$
which are indicator variables $\omega \in [0,1]$. Our base case model is
$\omega_1 = \omega_2 = 0$ where only the direct effects from $DC$ including the controls
is captured. Models 2 and 3 captures the mediation effect of $CAP$ and $COG$ respectively
($\omega_1$ or $\omega_2 = 1$). Our final model captures the effect of both mediators
($\omega_1 = \omega_2 = 1$).

A final model includes an interaction term between $CAP_t$ and $COG_t$ to capture any
complementarity or substitution-effects these two may exhibit:
\begin{align}
  CA_{i,1}  = & \alpha_0 + \alpha_1 CA_{i,0} + \omega_1 \alpha_2 CAP_{i,1}
  + \omega_2 \alpha_3 COG_{i,1} + \alpha_{4} DC_{i,0} +  \\ \nonumber
  &\alpha_5 CAP_{i,1} \times  COG_{i,1} + \mathbf{\Gamma}_m\mathbf{C}^m_{i,1} + \epsilon_i
\end{align}


\section*{Results}
The main results of our analysis are depicted both in the graphical path diagrams (figure
\ref{fig:med4}) and the regression table in table \ref{tab:res}. The use of PLS-SEM has
increased substantially in the realm of marketing \citep{hult2018} and strategic management
\citep{Hair2012,Valaei2017} and is the topic of special issues in leading journals \citep{Hair2020} (e.g. Journal of
Business Research call for papers in 2020). In this tradition we aim to present thorough
analysis of our models and largely follows the recommendations made in recent
xcontributions \citep{Hair2019}.

The empirical models are presented from left to right starting with the direct model. The
main effect from $DC_0$ to $CA_1$, controlled for $CA_0$, firm size, age, profit margin
and assets, is positive and significant at the onset. Increasingly, by adding mediating
variables as suggested by the theory, this effect diminishes to an extent suggesting that
partial or full mediation is taking place. The effect decreases from 0.25 to 0.08 and
turns insignificant when adding the two mediators. This suggests a full mediation. The
mediating variables $CAP_1$ and $COG_1$ are both positive and significantly related to
$CA_1$. This suggests that the effect between dynamic capabilities and competitive
advantage is mediated by these two factors separately, as well as together. These findings
both support H1 and H2. Moreover, the model becomes increasingly better both in terms if
explanatory power ($R^2$) and predictive power ($Q^2$).

When adding the interaction term $CAP_1 \times COG_1$ we get a significant effect. this is
compatible with the idea of behavioral ($CAP_t$) and non-behavioral ($COG_t$) objects of
change being complementary. This lends support to H3.

Our results indicate that between 37\% and 48\% of the variation in $CA_1$ can be
explained by our model, whereas 19\% and 18\% of the variation in our mediators $CAP_1$
and $COG_t$ respectively is explained.


\begin{table}[t]
  \caption{PLS-SEM results for key constructs}
\label{tab:res}
  \scriptsize






\begin{tabular}{l*{5}{l}}

  \hline                                                                                                                                                                                \\[-0.5em]
                           & (1)                        & (2)                           & (3)                           & (4)                        & (5)                              \\ 
  %                        & \multicolumn{1}{c}{Direct} & \multicolumn{1}{c}{CAP model} & \multicolumn{1}{c}{COG model} & \multicolumn{1}{c}{BOTH model}                                \\
  \hline                                                                                                                                                                                \\
  \rowcolor{Gray}
                           & $\mathbf{CA_t}$            & $\mathbf{CA_t}$               & $\mathbf{CA_t}$               & $\mathbf{CA_t}$            & $\mathbf{CA_t}$                  \\     
  $DC_{t-1}$               & 0.246             & 0.134                & 0.126                & 0.083             & 0.061                  \\
                           & (5.52)  & (2.72)     & (2.49)     & (1.6)  & (1.12)       \\ [1em]
  $CAP_{t}$                &                            & 0.382                &                               & 0.324             & 0.331                  \\
                           &                            & (7.65)     &                               & (6.38)  & (7.24)       \\[1em]
  $COG_{t}$                &                            &                               & 0.301                & 0.188             & 0.225                  \\
                           &                            &                               & (5.27)     & (3.11)  & (3.66)       \\ [1em]
    $CAP_{t} \times COG_t$ &                            &                               &                               &                            & 0.093                  \\
                           &                            &                               &                               &                            & (2.04)       \\ [1em]
    \rowcolor{Gray}
                           & $\mathbf{}$                & $\mathbf{CAP_t}$              & $\mathbf{}$                   & $\mathbf{CAP_t}$           & $\mathbf{CAP_t}$                 \\
  $DC_{t-1}$               &                            & 0.237                &                               & 0.224             & 0.206                  \\ 
                           &                            & (4.12)     &                               & (3.38)  & (3.17)       \\[1em]
    \rowcolor{Gray}
                           & $\mathbf{}$                & $\mathbf{}$                   & $\mathbf{COG_t}$              & $\mathbf{COG_t}$           & $\mathbf{COG_t}$                 \\
  $DC_{t-1}$               &                            &                               & 0.325                & 0.331             & 0.329                  \\
                           &                            &                               & (5.23)    & (5.29) & (5.31)       \\  \\[-0.5em]
  \hline                                                                                                                                                                                \\[-0.5em]
  CA controls              & YES                        & YES                           & YES                           & YES                        & YES                              \\
  CAP controls             &                            & YES                           &                               & YES                        & YES                              \\
  COG controls             &                            &                               & YES                           & YES                        & YES                              \\
  $R^2$                    & 0.3   & 0.44      & 0.37      & 0.47   & 0.48                             \\
  $Q^2$                    & 0.2   & 0.31      & 0.25      & 0.32   & 0.31                             \\
 % GOF                      & round(r[13,2],2)   & round(r[13,3],2)      & round(r[13,4],2)      & round(r[13,5],2)   &                                  \\
 % BIC                      & round(r[14,2],1)   & round(r[14,3],1)      & round(r[14,4],1)      & round(r[14,5],1)   & round((r[14,5]*0.996),1) \\
  N                        & 240   & 232      & 240      & 232   & 232                              \\
  \hline
 % \multicolumn{4}{l}{\tiny Standardized coefficients. T-values in parantheses}                                                                                                         \\
\end{tabular}
\begin{tablenotes}

      \tiny
      \item Standard-errors estimated using bootstrapping with 10 000 simulations. \vspace{-1em}
      \item Firm level control variables include firm size, firm age,
        profit margin, firm assets.
    \end{tablenotes}
  \end{table}


\subsection*{Robustness and assumption validation}
The results are subject to some robustness checks not directly reported in the paper,
albeit a summary is available in the appendix. First, controlling for functional form
using a Ramsay RESET test we find support for a linear relationship between our
constructs. Second, we ran the a Variance Inflation Test (VIF) to control for
multicolinearity. None of the VIF values exceeded the suggested level indicating
multicolinearity problems. Third, we ran two separate models where we (1) included other
measures of $DC$ (using the original measure of $DC_0$), and (2) added more control
variables to which decreased the sample size (from $N=238$ to $N=205$).

%%Appendix 2
%%presents the results from all these robustness checks and assumption validations.


%% Second, running a Bayesian Copula approach, we assess the problem of
%% endogeneity in the model stemming from causes such as omitted variable bias or
%% simultaneity. The approach explicitly models the correlation between the proposed
%% endogenous variable ant the error term by means of a copula. When including a copula in
%% the estimation as a correction for endogeneity we can compare changes in the estimated
%% coefficients in terms of how they change. We tested this approach on all our three main
%% independent variables ($CAP_1$, $COG_1$ and $DC_0$) and found no support for endogeneity
%% problems in our model. However, $CAP_1$ did not fully satisfy the non-normality assumption
%% of this approach as tested using a Kolmogorov-Smirnoff test, so we cannot fully conclude
%% that endogenous effects may be at play in the model. However, the normality test did not
%% establish normality, so we believe the Bayesian Copula approach was rather efficient for
%% our purpose. Consequently, we do not believe that we have endogeneity problems in our
%% model.
%% introduced the Gaussian copula approach, which controls for endogeneity by directly
%% modeling the correlation between the endogenous variable and the error term by means of a
%% copula Model estimation in PLS-SEM draws on a three-stage approach that belongs to the
%% family of (alternating) least squares algorithms (Mateos-Aparicio 2011). Stage 1 of the
%% PLS-SEM algorithm iteratively determines indicator weights, composite scores, and path
%% coefficients. The algorithm converges when the indicator weights change only marginally
%% from one iteration to the next. Stages 2 and 3 use the final composite scores from Stage 1
%% as input for a series of ordinary least squares regressions. These regressions produce the
%% final indicator loadings, indicator weights, and path coefficients, as well as related
%% elements, such as the indirect and total effects, and R2 values (e.g., Sarstedt et
%% al. 2017a). With Simply put, the copula connects the marginal distributions of two or more
%% variables that follow any distribution (e.g., normal, non-normal). Park and Gupta (2012)
%% add a copula term to the model that represents the correlation between the endogenous
%% variable and the error term. By including this term, the effect of the endogenous
%% regressor can be estimated consistently

\begin{figure}


\usetikzlibrary{arrows.meta}
% \begin{minipage}{.48\textwidth}
\begin{tabular}{cc}
  % \subcaption{DIRECT model}
 
  
 %\end{minipage}
 %\begin{minipage}{.48\textwidth}
   % \subcaption{CAP model}
  \begin{tikzpicture}
   
    \tikzstyle{mynode}=[circle,draw,thin, draw=black, fill=white, minimum
    size=12mm,inner sep=0pt, align=center]

    \tikzstyle{mynode2}=[circle,draw, ultra thin, draw=gray,text=gray, fill=white, minimum
    size=12mm,inner sep=0pt, align=center]

    \tikzstyle{gof}=[draw, ultra thin, draw=none, minimum size=18mm,inner sep=0pt, align=center]
        
        
    \node[mynode] (m){$CAP_t$};
    \node[mynode,below left=of m,yshift=0.5cm,xshift=-0.5cm](a) {$DC_{t-1}$};
    \node[mynode,below right=of m,yshift=0.5cm,xshift=0.5cm](b) {$CA_{t}$};
    \node[mynode2,below of=m,yshift=-1.5cm](c) {$COG_{t}$};
    \node[gof,below of=c,yshift=-0cm](gof) {$R^2= 0.44$, $Q^2= 0.31$};
    
    \draw[-latex, line width=0.3mm] (a.east) -- node[text width=0.8cm,font=\scriptsize, align=center,
    fill=white] {$0.134^{*}$} (b.west);
    
    \draw[-latex, line width=0.3mm] (a.north) -- node[text width=0.8cm,font=\footnotesize, align=center, fill=white] {$0.237^{*}$} (m.west);
    \draw[ -latex, line width=0.3mm] (m.east) -- node[text
    width=0.8cm,font=\footnotesize,align=center, fill=white] {$0.382^{*}$} (b.north);

    \draw[gray,-latex] (a.south) -- node[text width=1.3cm,font=\scriptsize, align=left, fill=none] {} (c.west);
    \draw[gray, -latex,line ] (c.east) -- node[text
    width=1.3cm,font=\scriptsize,align=right, fill=none] {} (b.south);
    
    
   \end{tikzpicture}
 %\end{minipage}
 %\begin{minipage}{.48\textwidth}
 %  \subcaption{COG model}
    &
  \begin{tikzpicture}
   
    \tikzstyle{mynode}=[circle,draw, thin, draw=black, fill=white, minimum
    size=12mm,inner sep=0pt, align=center]

    \tikzstyle{mynode2}=[circle,draw, ultra thin, draw=gray,text=gray, fill=white, minimum
    size=12mm,inner sep=0pt, align=center]

    \tikzstyle{gof}=[draw, ultra thin, draw=none, minimum size=18mm,inner sep=0pt, align=center]
        
        
    \node[mynode2] (m){$CAP_t$};
    \node[mynode,below left=of m,yshift=0.5cm,xshift=-0.5cm](a) {$DC_{t-1}$};
    \node[mynode,below right=of m,yshift=0.5cm,xshift=0.5cm](b) {$CA_{t}$};
    \node[mynode,below of=m,yshift=-1.5cm](c) {$COG_{t}$};
    \node[gof,below of=c,yshift=-0cm](gof) {$R^2= 0.37$, $Q^2= 0.25$};
    
    \draw[-latex, line width=0.3mm] (a.east) -- node[text width=0.8cm,font=\scriptsize, align=center,
    fill=white] {$0.126^{*}$} (b.west);
    
    \draw[gray,-latex] (a.north) -- node[text width=0.8cm,font=\scriptsize, align=left, fill=none] {} (m.west);
    \draw[gray, -latex,line ] (m.east) -- node[text
    width=0.8cm,font=\scriptsize,align=right, fill=none] {} (b.north);

    \draw[-latex, line width=0.3mm] (a.south) -- node[text width=0.8cm,font=\scriptsize, align=center, fill=white] {$0.325^{*}$} (c.west);
    \draw[-latex, line width=0.3mm] (c.east) -- node[text
    width=0.8 cm,font=\scriptsize,align=center, fill=white] {$0.301^{*}$} (b.south);
    
    
   \end{tikzpicture}
 %\end{minipage}
 %\begin{minipage}{.48\textwidth}
 %  \subcaption{FULL model}
  \\
  \begin{tikzpicture}
   
    \tikzstyle{mynode}=[circle,draw, thin, draw=black, fill=white, minimum
    size=12mm,inner sep=0pt, align=center]

    \tikzstyle{mynode2}=[circle,draw, ultra thin, draw=gray,text=gray, fill=white, minimum
    size=12mm,inner sep=0pt, align=center]

    \tikzstyle{gof}=[draw, ultra thin, draw=none, minimum size=18mm,inner sep=0pt, align=center]
        
        
    \node[mynode] (m){$CAP_t$};
    \node[mynode,below left=of m,yshift=0.5cm,xshift=-0.5cm](a) {$DC_{t-1}$};
    \node[mynode,below right=of m,yshift=0.5cm,xshift=0.5cm](b) {$CA_{t}$};
    \node[mynode,below of=m,yshift=-1.5cm](c) {$COG_{t}$};
    \node[gof,below of=c,yshift=-0cm](gof) {$R^2= 0.47$, $Q^2= 0.32$};
    
    \draw[-latex,line width=0.3mm] (a.east) -- node[text width=0.8cm,font=\scriptsize, align=center,
    fill=white] {$0.083$} (b.west);
    
    \draw[-latex,line width=0.3mm] (a.north) -- node[text width=0.8cm,font=\scriptsize, align=left, fill=white] {$0.224^{*}$} (m.west);
    \draw[ -latex,line width=0.3mm ] (m.east) -- node[text
    width=0.8cm,font=\scriptsize,align=right, fill=white] {$0.324^{*}$} (b.north);

    \draw[-latex,line width=0.3mm] (a.south) -- node[text width=0.8cm,font=\scriptsize, align=left, fill=white] {$0.331^{*}$} (c.west);
    \draw[-latex,line width=0.3mm ] (c.east) -- node[text
    width=0.8cm,font=\scriptsize,align=right, fill=white] {$0.188^{*}$} (b.south);
    
    
   \end{tikzpicture}
    &
    \begin{tikzpicture}
   
    \tikzstyle{mynode}=[circle,draw, thin, draw=black, fill=white, minimum
    size=12mm,inner sep=0pt, align=center]

    \tikzstyle{mynode2}=[circle,draw, ultra thin, draw=gray,text=gray, fill=white, minimum
    size=12mm,inner sep=0pt, align=center]

    \tikzstyle{gof}=[draw, ultra thin, draw=none, minimum size=18mm,inner sep=0pt, align=center]
        
        
    \node[mynode] (m){$CAP_t$};
    \node[mynode,below left=of m,yshift=0.5cm,xshift=-0.5cm](a) {$DC_{t-1}$};
    \node[mynode,below right=of m,yshift=0.5cm,xshift=0.5cm](b) {$CA_{t}$};
    \coordinate (CAf) at ([xshift=-1.25cm]b.west);
    \node[mynode,below of=m,yshift=-1.5cm](c) {$COG_{t}$};
    \node[gof,below of=c,yshift=-0cm](gof) {$R^2= 0.48$, $Q^2= 0.31$};
    
    \draw[-latex,line width=0.3mm] (a.south east) -- node[above,text width=0.8cm,font=\scriptsize, align=left,
    fill=white] {$0.061$} (b.south west);
    
    \draw[-latex,line width=0.3mm] (a.north) -- node[text width=0.8cm,font=\scriptsize, align=left, fill=white] {$0.206^{*}$} (m.west);
    \draw[ -latex,line width=0.3mm ] (m.east) -- node[text
    width=0.8cm,font=\scriptsize,align=right, fill=white] {$0.331^{*}$} (b.north);

    \draw[-latex,line width=0.3mm] (a.south) -- node[text width=0.8cm,font=\scriptsize, align=left, fill=white] {$0.329^{*}$} (c.west);
    \draw[-latex,line width=0.3mm ] (c.east) -- node[text
    width=0.8cm,font=\scriptsize,align=right, fill=white] {$0.225^{*}$} (b.south);
    
     \draw[line width=0.3mm] (m.south east) to[bend left=20] node[text width=0.8cm,font=\scriptsize, align=center,
    fill=none] {} (c.north east);
    
    \draw[{Circle[black]}-latex,line width=0.3mm] (CAf.east) -- node[above, text width=0.2cm,font=\scriptsize, align=center,
    fill=white] {$0.093^{*}$} (b.west);   
    
   \end{tikzpicture}
% \end{minipage}
\end{tabular}
\caption{Mediation analysis - graphical representation}
\label{fig:med4}
\end{figure}


\section*{Discussion}
The aim of this paper is to theorize about how non-behavioral objects of change work as a
mechanism through which dynamic capabilities influence competitive advantage. Our results
indicate that both behavioral (\emph{operating capabilities}) and non-behavioral
(\emph{strategic cognition}) mediates the effect from dynamic capabilities on competitive
advantage. The first mechanism, dubbed 'conducting' in reference to the well established
terminology of 'resource orchestration' \citep{Sirmon2011,Helfat2007}, shows a robust and
strong function as a mediator in line with our first hypothesis. When firms exhibit strong
dynamic capabilities they are able to adapt their operating capabilites to stay
competitive. We hypothesized that this relationship would be positive and significant. Our
results indicated exactly this relationship and thus lending support to hour hypothesis.
In our example context of Microsoft Nadella did several investments in capabilities and
resources as a response to Microsofts dynamic capabilities execution. For example, he
acquired important capabilities from M\&A efforts, and upgraded the competencies regarding
cloud business and big data \citep{Nadella2017}. This should not be surprising given the
vast literature on indirect effects of dynamic capabilities
\citep{Schilke2018,Eriksson2014} supported by empirical and theoretical research
\citep{Pezeshkan2016b,Fainshmidt2016a,Protogerou2012,Karimi2015}. However, it adds to this
ongoing effort of investigating mechanisms through which dynamic capabilities work which
is called for in the literature \citep{Schilke2018}.

In this vein, our second hypothesis, and the main contribution of this paper, is much more
novel in its characteristics. Building on the important work of \cite{Verona2011} in what
they call adding a 'human side to dynamic capabilities' we suggest that non-behavioral
objects of action exists in parallel with behavioral ones. This view holds that cognition
makes a difference in creating strategic change and should thus be a part of a more
'holistic' approach to dynamic capabilities theory (ibid). The notion of cognition in
general plays an important role in determining 'the way we do things'
\citep{Vince2011}. The idea of 'strategic cognition' seems to be a way of safeguarding
that organization members buy into the needed strategic change identified by execution of
dynamic capabilities. Indeed, this is also what we find in our results. Our
operationalization of 'strategic cognition' as \emph{proactiveness} makes for a proper
mediator in the focal relationship between dynamic capabilities and competitive
advantage. This is the path we have dubbed 'coaching' where dynamic capabilities rather
than orchestrating resources enables the organization to fully grasp and adapt to the
proposed strategic change. In our Microsoft example this was exactly where Nadella put the
most emphasis. By instilling a growth mindset Microsoft became a more proactive company,
but the triggering mechanism was the execution of dynamic capabilities through sensing and
seizing the opportunities they saw.

Our empirical results support the idea that \emph{operating capabilities}) and
\emph{strategic cognition} being distinctly different mechanisms through which dynamic
capabilities influence competitive advantage. This suggests that dynamic capabilities have
two roles working in tandem, both being a 'conductor' of capabilities, and a 'coach' of
cognition at the same time. Moreover, these two types of objects of change, as thoroughly
discussed by \cite{Verona2011}, are not independent. They work in tandem in that you are
not able create strategic change without adapting your capabilities for the change,
meanwhile exercising the right interpretation, understanding and attitude towards
change. Our third hypothesis considers strategic cognition in tandem with relevant and
strong capabilities and hold that they would reinforce each other. This has also been
supported in previous research where firm success has been shown to relate to the
interconnection between resources and capabilities, and culture and attitude
\citep{Verona2003}, and high levels of social capital \citep{Blyler2003}. Moreover, the
lack of one (e.g. strategic cognition) can lead to unfavorable outcomes even though
capabilites and resources are aligned for the needed strategic change
\citep{Tripsas2000}. This is also one of the key lessons from the Microsoft case. Our
empirical results do indeed support this. \emph{Operating capabilities}) and
\emph{strategic cognition} are complementary to each other in influencing competitive
advantage meaning that businesses will not succeed properly without being conscious of
both.



\section*{Concluding remarks}

We started this inquire with the aim to shed light on non-behavioral objects of change as
an alternative mechanism between dynamic capabilities and competitive advantage. Inspired
by the novelty of insights into the 'human side of dynamic capabilities'
\citep{Verona2011,Zollo2016}, the forceful transformation of Microsoft (Nadella), and the
call to investigate new mechanisms (mediators) in dynamic capabilities
\citep{Schilke2018}, we set out to analyze the role of these non-behavioral objects of
change. Indeed, both research and practical interest in dynamic capabilities largely stems
from the promise of delivering competitive advantage - the 'holy grail' of strategic
management \citep{Helfat2007,Schilke2018,Peteraf2013}. Thus, really understanding the
mechanisms \emph{through} which dynamic capabilities can influence competitive advantage,
is of crucial importance.

Using recent contributions on the role of cognition in strategy
\citep{Gavetti2012,Zollo2016,Verona2011} we hypothesized about the role strategic
cognition, defined as 'conceptual structures in the minds of the individuals that
encapsulate a shared understanding of the reality the individuals face with respect to
the strategic goal of the organization'.

Using a survey of $262$ Norwegian technology firms we constructed latent variables
capturing the focal constructs of dynamic capabilities theory. Using PLS-SEM analysis we
were able to analyze the relationships between these constructs, specifically the role of
\emph{operating capabilities} and \emph{strategic cognition} as measured by capabilities
for daily tasks, and proactiveness respectively. We found that these two are separate
constructs both acting as mediators between dynamic capabilities and competitive
advantage. Moreover, an interaction analysis also suggest that they are indeed
complementary to each other.

We have aimed at using this holistic approach to dynamic capabilities theory to explain
firm competitive advantage - the 'holy grail' of strategic management \citep{Schilke2018}.
Specifically we have tried to push the theory forward in understanding of \emph{how}
dynamic capabilities work to influence firms' competitive advantage. Our contributions
have been threefold. First, we clarified the theoretical role of non-behavioral objects of
change and suggested a possible empirical operationalization through the concept of
\emph{proactiveness}. This has implications for how we operationalize important mediators
in the $DC-CA$ nexus.

Second, we suggested and demonstrated two different expressions of dynamic capabilities
(i.e. {\bf conductor} and {\bf coach}) as well as their workings in the dynamic
capabilities-competitive advantage nexus. This insight adds to the nascent work of
bringing cognition into broader strategic management field in general
\citep{Gavetti2012,Gavetti2005a} and dynamic capabilities in particular
\citep{Helfat2015,Arndt2018}. In particular, the evolutionary and resource based roots of
$DC$ theory has to a large extent been dominating the field \citep{Arndt2018,Peteraf2013}
leaving an important side of cognition \citep{Verona2011} and heuristics \citep{Arndt2015}
less explored. Particularly, the call for exploring new mediators of the $DC-CA$ nexus
\citep{Schilke2018} is particularly relevant to moving $DC$ theory forward. Our
conceptualization is thus an important contribution to this work. Bringing the notion of
simultanously separate and interpendent expressions of $DC$ has implications for
strategizing and decisions to pursue capability aquisition and development. As with the
case of Microsoft, our data shows that simultanously maintaining 'strategic cognition'
adds to outcome of the strategic change. Thus, managers who underestimate these effects
may very well end up delivering less value than they otherwise could.

Third, we utilized time-lagged data in conjunction
with advances in empirical estimations of multiple simultaneous mediators. In doing these
three we believe we have helped to push our understanding of important mechanisms in the
dynamic capabilities theory.

This paper has several limitations however. First, it has been hard to capture the full
extent of the focal constructs empirically. We have had to settle for few items to capture
the full magnitude of the mechanisms. Future research should focus on establishing more
comprehensive operationalization of key constructs in order to better test the
theory. Second, we have leaned a lot on one case study to motivate our work. Although
Microsoft is an interesting and important case, future research should include more case
studies with more in-depth analysis to really get under the proposed mechanisms. Third,
our sample size is on the smaller size. Granted, PLS-SEM is able to capture rather
consistent estimates on small samples, larger samples would enable researchers to extend
the analysis further. More industries and countries should also be included in the
analysis, and more observations over time would be helpful in analyzing more causal claims
in the theory. Forth, we have examined two particular types of behavioral and
non-behavioral objects of change. Thus, future research should examine other types in
order to develop e more elaborate and holistic understanding of how behavioral and
nonbehavioral objects of change influence competitive advantage in the theory of DC.

These limitations aside, we believe that this paper has helped to push the theory in a
more 'holistic' direction as suggested by \cite{Verona2011} and hence answering the call
for investigating more mechanisms in the dynamic capabilities theory \citep{Schilke2018}.


%\section*{Bibliography}


%\singlespacing
%\bibliography{Dissertation_clean}


\newpage
\section*{Appendix}

\begin{table}[h]
  \label{tab:items}
    \caption{Core constructs and their measurement items. Factor
      loadings in parantheses}



\scriptsize
\singlespacing
\setlist[itemize]{leftmargin=*}



\begin{tabular}{p{2cm} c p{9cm}}

\textbf{Construct}     & \textbf{Rho}               & \textbf{Items}\\ \toprule%
  \textbf{Dynamic} \\ \textbf{Capabilities} &\textbf{$0.89$} &\vspace{-5mm} 
                                                             \begin{itemize} 
                                                               \setlength\itemsep{-0.2em} 
                                                             \item Management receive frequent information about the employee's newly gained experiences ($0.69$)
                                                             \item Knowledge and experiences are shared a cross teams in the firm ($0.77$)
                                                             \item Knowledge and experiences are shared a cross teams in the firm ($0.76$)
                                                             \item Positive and negative experiences are shared between employees ($0.82$)
                                                             \item The employees are involved in discussions where the way of doing business is questioned ($0.79$)
                                                             \item The employees are encouraged to engage in critical discussions on how the company is conducting its business ($0.62$)
                                                             \item We conduct evaluations of "what has worked and what has not worked" in regards to larger projects etc ($0.61$)
                                                             \end{itemize} \\ \midrule%
 \textbf{Operating}                                                                            \\ \textbf{Capabilities} & \textbf{$0.84$}} & \vspace{-5mm} 
                                                               \begin{itemize} \setlength\itemsep{-0.2em} 
                                                               \item The firm has better marketing competence than the competitors ($0.78$)
                                                               \item Compared to our competitors our firm is particularly skilled in customer service ($0.71$)
                                                               \itemCompared to our competitors our firm is particularly skilled in management and operations  ($0.86$)
                                                               \end{itemize} \\  \midrule%
  
   \textbf{Strategic} \\ \textbf{Cognition} & \textbf{$0.82$} & \vspace{-5mm} 
                                                               \begin{itemize} \setlength\itemsep{-0.2em} 
                                                               \item Compared to our competitiors our firm typically refrain from action until competitors move, and then answer ($0.89$)
                                                               \item Compared to our competitors we are often first movers in introducing new products, services, administrative routines, production methods etc ($0.87$)
                                                               \item We relate to our competitors by adopting a "go get them" attitude. We are gonna win this ($0.53$)
                                                               \end{itemize} \\  \midrule%
  
   \textbf{Competitive} \\ \textbf{Advantage} & \textbf{$0.89$} & \vspace{-5mm} 
                                                               \begin{itemize} \setlength\itemsep{-0.2em} 
                                                               \item The firm has better financial results than our competitors ($0.85$)
                                                               \item The firm has a stronger revenue growth than our competitors ($0.87$)
                                                               \item The firm has a larger market share than our competitors ($0.87$)
                                                               \end{itemize} \\  \bottomrule%
  
\end{tabular}
    \begin{tablenotes}
      \tiny
      \item Dillon-Goldstein Rho is recommended as reliability measure
        for SEM-PLS \citep{Hair2019}.
    \end{tablenotes}
  \end{table}



\end{document}


