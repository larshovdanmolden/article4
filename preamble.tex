\usepackage{lineno,hyperref,float}
\usepackage{amsmath,natbib,bm}
\usepackage{geometry,array,setspace,lipsum,nccmath}
\usepackage{amsfonts}
\usepackage{booktabs}
\usepackage{siunitx}
\usepackage{rotating}
\usepackage{hypcap}
\usepackage{adjustbox}
\usepackage{pdflscape}
\setstretch{1.0}
\usepackage{color}
\usepackage{framed}
\usepackage{comment}
%\definecolor{shadecolor}{gray}{0.875}
\usepackage{caption}
%\captionsetup[table]{skip=8pt}
\usepackage{enumitem}
\usepackage{longtable}
% Table float box with bottom caption, box width adjusted to content
\usepackage{afterpage}
\usepackage{blindtext}
\usepackage{alltt}
\usepackage{graphicx}
\usepackage{graphics}
\usepackage{tikz}
\usepackage{tcolorbox}

\usetikzlibrary{shapes.geometric, arrows}
\usetikzlibrary{arrows.meta}
\usepackage{tikzscale}
\tikzset{every picture/.style={font issue=\footnotesize},
	font issue/.style={execute at begin picture={#1\selectfont}}
}


\renewcommand\labelitemi{--}
\tolerance=1600


\hypersetup{
  colorlinks=true,
  linkcolor=blue,    % color of internal links
  citecolor=blue,    % color of links to bibliography
  urlcolor=blue,     % color of external links
  allcolors=blue,
  bookmarksopen=true,
  pdfdisplaydoctitle=true
}
%\usepackage[doublespacing]{setspace}
%%\renewcommand\arraystretch{1.3}

\makeatletter
\@addtoreset{section}{part}
\makeatother
%\setcounter{secnumdepth}{0} % sections are level 1
\newcommand\T{\rule{0pt}{2.6ex}}       % Top strut
\newcommand\B{\rule[-1.2ex]{0pt}{0pt}} % Bottom strut
%\renewcommand{\familydefault}{\sfdefault}

\makeatletter
\renewcommand{\todo}[2][]{%
    \@todo[caption={#2}, #1]{\begin{spacing}{0.5}#2\end{spacing}}%
} 
\makeatother 

\usepackage{natbib}
%\bibpunct{(}{)}{;}{a}{,}{,}
%\usepackage{biblatex}
%\addbibresource{test}
\usepackage{tocloft}

%\usepackage{harvard}
\modulolinenumbers[5]

\journal{TBD}

%%APA style
\bibliographystyle{model5-names}\biboptions{authoryear}
%\specialcomment{answer}{\begin{shaded}}{\end{shaded}}

%% `Elsevier LaTeX' style
%\bibliographystyle{elsarticle-num}
%%%%%%%%%%%%%%%%%%%%%%%
\setlength\parindent{0pt}
\setlength{\parskip}{6pt}


