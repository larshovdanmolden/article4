The other path, the market path, is characerized by smaller routines and practices that
themselves are not satisfying the VRIO condition (preciclsy as argued by EM). However, in
factor markets  
simple routines and complex processes can lead to comp


Market path: simple rules guiding actions in markets for aquistion of assets. Not
themselves VRIO, but taken together with the assets they aquire more idiosyncratic to the
firm. These simple rules are routines and they corresponds to rapid changes and the rear
wheel. mor profound in dynamic environments
Eisenhardt and Sull (2001) credited its prowess in this arena to the simple rules that Cisco has developed to guide its acquisition process. Cisco

Evolutionary path: More complex rules and process working to change the operational
routines of the firm. These are themselves idiosyncratic and VRIO (vrio stems from the DC
level and hence generates path dependent routines at theOR level. Corresponds to more
gradual change. the front crank / more gradual environments. More complex processes need
to work thorugh a mediation (OR). More simple rules works more directly (unobserved)

\cite{DiStefano2014} propose a metaphore to think of these paths to competetive
advantage through the idea of the "organizational drivetrain". 

The interplay prposed by DiStefano 

We find that DC as routines to shape routines is most prevelant under moderate dynamism,
whereas DC as routines to capture resources in factor markets are more prevelant under
dynamic environments. 

Under certain conditions the DC operates through changing underlying routines. OTher times
they work through factor markets such as EM and the Cisco case. 

Routinization indicates the extent to which organizational pro- cesses are stable and repetitive (Nelson &Winter, 1982) and constitutes an important revelatory access to the nature of dynamic capabilities (Barney & Felin, 2013). The sub-stream around Teece et al.’s (1997) seminal work largely argues that dynamic capabilities rely on highly rou- tinized processes, whereas the sub-stream around Eisenhardt and Martin's (2000) conceptualization argues for reduced routinization (Peteraf et al., 2013; Schreyögg & Kliesch-Eberl, 2007)


Start with two contradictory definitions:
Kale and Singh (2007, p. 982) ex- plained that “dynamic capability refers to the capac- ity of an organization to purposefully create, extend, or modify its resources or skills.”

Zollo and Winter (2002, p. 340) explained that “a dynamic capability is a learned and stable pattern of collective activity through which organizations systematically generate and modify operating rou- tines for improved effectiveness.”



One contentious issue surrounding the DC theory is the extent to which differing
views on the nature of dynamic capabilities \citep{Peteraf2013}. One strand of
contributions sees DC as routines that are simple in nature and more as best practices to
consider (EM). Thus, dynamic capabilities are more like behavioral patterns and
less idiosyncratic to the firm and will consequently not be able to satisfy the VRIO
condition for competitive advantage to emerge. These routines are  

On the other hand, a stream of contributions see dynamic capabilities as evolutionary and
more complex routines (Ref) or processes (ref) set to change underlying capabilities in
the face of changing conditions (TPS).  

CA RESULT OF DYNAIMC BUNDLE
to the socially complex and hard-to- imitate dynamic bundle of resources and capabili- ties (Peteraf et al., 2013) whose



Market path: simple rules guiding actions in markets for aquistion of assets. Not
themselves VRIO, but taken together with the assets they aquire more idiosyncratic to the
firm. These simple rules are routines and they corresponds to rapid changes and the rear
wheel. mor profound in dynamic environments
Eisenhardt and Sull (2001) credited its prowess in this arena to the simple rules that Cisco has developed to guide its acquisition process. Cisco

Evolutionary path: More complex rules and process working to change the operational
routines of the firm. These are themselves idiosyncratic and VRIO (vrio stems from the DC
level and hence generates path dependent routines at theOR level. Corresponds to more
gradual change. the front crank / more gradual environments. More complex processes need
to work thorugh a mediation (OR). More simple rules works more directly (unobserved)

On the notion of observability we can observe changes in underlying routines (OR) whereas
it is harder to observe routines generating opportunities in the market (dircet route)


Eisenhardt and Sull (2001) credited its prowess in this arena to the simple rules that
Cisco has developed to guide its acquisition process. Cisco

As an answer to the call for more nuanced theorizing on the differing role of dynamic
capabilities in shaping competitive advantage, and their contingencies
\citep{Peteraf2013}, \cite{DiStefano2014} propose theoretical model distinguishing between
rapid and routinized changes on the one hand, and slower, less routinized changes on the
other. Using the "organizational drive train" metaphor


The nature of DC: 
The fundamental difference between these two conceptions is related to the degree of observabil- ity. Action that is latent cannot be observed until called into use, while constituent elements have a more concrete and observable form (Helfat, Finkel- stein, Mitchell, Peteraf, Singh, & Winter, 2007). This has implications for the empirical identifica- tion of dynamic capabilities and suggests some of the challenges involved

DC as enabling device (latent) and as process (observable THROUGH changes in OR). 



%Latent action / process vs routine - can be both??
 %% Interestingly, the distinction be- tween dynamic capabilities as an ability versus dy- namic capabilities as a process dates back to the two seminal manuscripts, with TPS advocating the former position and EM supporting the latter. 

DRIVE TRAIN EXPLAINED


%% Given the extent and import of these differences
%% in the seminal views, how then can we bring them together fruitfully? Our answer is that we can do so by broadening our perspective to see the two views as each focused on a different part of a larger, interconnected, and more fully dynamic system.

We claoim, contrary to Di Stefano, that the two mechanisms in the drive train yields
differing results. The simple rules tend to only yield CA when they can operate in factor
markets to capture unique compistiosn of resources - ie.e their CA stems from their
ability to systematiclally find resrources in factor markets needed to adapt to
change. these resoures nor the routines themselves are sufficiently VRIO, but taken
together they form a powerful lever for change (increaseing or decreasing power from the
crankset).



OTHER METAPHOR: HYBRID CAR
Running electric motor in the gradual environment building speed continuosly (changing
rountines and hence moving more like a process.


ROLE OF DYN:
Last, our conception of an organizational drive-
train also suggests a resolution to the debate over whether dynamic capabilities in high-velocity en- vironments can go beyond managing change to also provide a sustainable competitive advantage. In this respect, we note that even if specific simple rules are unstable and ephemeral, the system as a whole is not. Moreover, the interconnectedness, reliance on tacit knowledge, and complexity of a system involving a variety of moving parts suggest that the real source of sustainable competitive ad- vantage for an enterprise is the difficulty of imitat- ing and substituting for the entire dynamic bundle (Peterafthat the system represents.


Previous work has highlighted the role of higher-order capabilities in a way that is
consistent with the idea of a dynamic bundle \citep{Schilke2014a}. This concept of higher-r
The view of the organiza- tional drivetrain is also consistent with works that tend to conceive dynamic capabilities as higher- order capabilities because it helps with envisioning different types of linkages between higher-order capabilities and organizational routines (see Kahl, 2014 and Schilke, 2014).
