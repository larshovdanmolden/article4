




%Mindset induction effects on cognitive control: a neurobehavioral investigation.
%Schroder HS, Moran TP, Donnellan MB, Moser JS
%Biol Psychol. 2014 Dec; 103():27-37.

%12. Yeager D.S., Dweck C.S. Mindsets that promote resilience: When students believe that personal characteristics can be developed. Educ. Psych%ol. 2012;47:302–314


Proactiveness is an opportunity-seeking, forward-looking perspective characterized by the introduction of new products and ser- vices ahead of the competition and acting in anticipation of future demand.

Proacteivenss can create psychological ownership (siegel 
We build on these theoretical considerations by applying the concept of psychological ownership (Pierce et al., 2003), defined as ‘the state in which individuals feel as though the target of ownership or a piece of that target is “theirs” ’ (Pierce et al., 2003, p. 86)

Missing link between growth mindset and internal motivation
https://www.ncbi.nlm.nih.gov/pmc/articles/PMC5836039/

Growth mindset and individual proactivity
https://www.emerald.com/insight/content/doi/10.1108/CDI-11-2016-0194/full/html



To be clear, the importance of motivation and, in particular, its strategic relevance for
organi- zational performance and competitive advan- tage has surfaced in some recent
contributions in the strategic management field. Managerial rent theorists (Castanias &
Helfat, 1991, 2001), for example, submit that the availability of skills that have rent
generation potential is a neces- sary but not sufficient condition for rent gener-
ation. In so doing, they also point to the impor- tance of motivation for the realization
of any such potential. Similarly, Coff (1997) discusses a number of dilemmas firms face in
generating an advantage from human assets, and he points to the importance of governance
mechanisms to overcome them. Finally, Makadok (2003) calls for future research into the
genesis of competitive advantage in a way that combines competence- based issues with
questions of governance or motivation.

of the
company’s existence and what defi nes or limits its role(s) within its social and eco-
nomic environment (organizational identity)? What does it stand for (organizational val-
ues)? How are things done over here (organizational culture)?

include norms (leonard Barton 1992), avoidance of cognitive
inertia (Trpisas and Gavetti 2000)

organizational inertia
Second, 
The existence of layer upon layer of standard procedures, established capabilities, complementary assets, and/or admin- istrative routines can exacerbate decision-making biases against innovation. Incumbent enterprises,


Individuals are the unit of analysis embedding these abilities and if left up to
these individuals without a higher organizing framework, the firm will be very vulnerable
because of knowledge asymmetries and incentive problems (alchian and demsetz 1972) and
lead to sub optimal rent-appropriation on the part of the individual employee
\citep{Blyler2003}. A similar argument applies to the function of \emph{seizing} where the
organization is set to mobilize resources for the purpose of gaining from these
opportunities. In order for value to be captured from opportunities, a string of
conditions have to be met \citep{Teece2017}, but the evolution of a 'strong and
change-oriented organizational culture' \citep[p. 331]{Teece2014} is of pivotal
importance. Alas the sensing process requires new knowledge from a wide range of sources
in the organization a lot of this tacit knowledge can lead to rent-appropriation
from information asymmetries if not sufficiently aligned with the
cognitive frames of the organization \citep{Blyler2003}. Furthermore, as cognition plays
an important role in the sensing and seizing of opportunities \citep{Helfat2015} and the
interpretation and understanding varies with individuals position within the
organization, then diverging interpretations stemming from misaligned cognitive frames may
lead to 

They maintain that social capital and notably individuals’ valuable internal and external social ties allow for information sharing, innovation and novel ways of thinking which in turn helps managers understand resource acquisition, integration and release


Although they relate the existence of social capital
as an antecedents to dynamic capabilities, the same argument can be applied to their
function; to be able to capture rents for the organization and not the individual a
certain level of social capital is needed and 


“firms would be unable to acquire, recombine, and release resources”
resources” (2003: 680) without the social capital of individuals. Hence need to build
social capital. Growth mindset - cognitive frames important to build social captail within
the organization?



One consequence is a ‘program persistence bias.’ Its corollary is various forms of
‘anti-innovation bias,’ including the ‘anti- cannibalization’ basis discussed in a later
section. Program persistence refers to the funding of pro- grams beyond what can be
sustained on the merits,

the emphasis is on the intermediary role of "asset
orchestration" \citep{Teece2012,Sirmon2011} as related to the "the selection, con-
figuration, alignment, and modification of tangible and intangible assets"
\citep[p.842]{Helfat2015}. This process is very much similar to the exploration and
exploitation of opportunities and we consider reconfiguration to come into the focus when
these processes are seen over time and in changing contexts. Hence we are limiting this
study to looking at the sensing and seizing class of dynamic capabilities.



However, in this endeavour
simply adatping resources is a necessary but not sufficient condition for obtaining the
needed strategic change.   

The functions of dynamic capabilities as highlighted above is thoroughly discussed in
extant literature (refs), but a quick recap is useful. 

DC geared towards creating new or reconfiguring old (exploration vs exploitation) but true
competetiveness over time stems from ambidexterity- link to research on cultural ambidecterity


To see how dynamic capabilities work to adapt cognitive frames in addition to operational
capabilities to influence competetive advantage, it is useful to turn to their
microfoundations \citep{Teece2007}. In this framework dynamic capabilities work through
three expressions. First, sensing seizing. more here..

Each of these dimensions relates to a function of cognitive frames within the firm beyond
the cognitive frames of the dynamic capability itself as explored by previous
contributions in general \citep{Teece2007} and towards management specifically
\citep{Helfat2015,Helfat2015a}. First, sensing opportunities in the market relates to
discovering market disequilibrium \citep{Kirzner1997} as well as upsetting existing
equilibria \citep{Schumpeter1934}. This process is very much at the core of dynamic
capabilities as a function of management, but its function calls for a very close
relationship with the rest of the organization both in terms of resource orchestration and
cognitive programming. The process of sensing 'involves understanding latent demand, the
structural evolution of industries and markets, and likely supplier and competitor
responses' \citep[p. 1322]{Teece2007}. Individuals are the unit of analysis embedding
these abilities and if left up to these individuals without a higher organizing
framework, the firm will be very vulnerable because of knowledge asymmetries and incentive
problems (alchian and demsetz 1972) and lead to sub optimal rent-appropriation on the part
of the individual employee \citep{Blyler2003}. 

A similar argument applies to the function of \emph{seizing} where the organization is set
to mobilize resources for the purpose of gaining from these opportunities. In order for
value to be captured from opportunities, a string of conditions have to be met
\citep{Teece2017}, but the evolution of a 'strong and change-oriented organizational
culture' \citep[p. 331]{Teece2014} is of pivotal importance. Alas the sensing process
requires new knowledge from a wide range of sources in the organization a lot of this
knowledge is tacit and can lead to rent-appropriation from information asymmetries
\citep{Blyler2003} if not sufficiently aligned with the cognitive frames of the
organization.

Finally, the \emph{transformation} function of dynamic capabilities are related to the
idea of continuously renewal of the resource base in order to stay competitive. This continuous
process requires the development and maintenance of a share cognitive frame such as a
culture, which is typically found in production processes such as LEAN and Six Sigma
(ref).     

Looking at dynamic capabilities through the functions of sensing, seizing and
transformation makes an obvious case for resource orchestration which is also highlighted
in the extant literature. Moreover, as discussed above, these functions are also requiring
a certain non-behavioral object of action in order to be fully efficient. To achieve a
competitive edge firms will have to be able to sense and seize, which require a certain
organizational cognitive frame to be efficient. 'Dynamic capabilities are about doing the
right things, at the right time, based on new product (and process) development, unique
managerial orchestration processes, a strong and change-oriented organizational culture'
\cite[p]{Teece2014}. 


It is argued that contextual ambidexterity is grounded in the type of organizational culture (Ghoshal and Bartlett 1994; Gibson and Birkinshaw 2004; Simsek et al. 2009) that promotes both creativity and discipline (Jelinek and Schoonhoven 1993), or both the presence of different knowledge and the integration of multiple perspectives to develop a cohesive point of view (Eisenhardt and Schoonhoven 1990).

ambidextrous organizational culture wang rafiq 2014


The enterprise will be vulner- able if the sensing, creative, and learning functions are left to the cognitive traits of a few individuals


Three types of managerial activities can make a capability dynamic: sensing (which means
identifying and assessing opportunities outside your company), seizing (mobilizing your
resources to capture value from those opportunities), and transforming (continuous
renewal).


Whereas ordinary capabilities are about doing
things right, dynamic capabilities are about doing the right things, at the right time, based on new product (and process) development, unique mana- gerial orchestration processes, a strong and change- oriented organizational culture, and a prescient assessment of the business environment and tech- nological opportunities.
teece 2014


The strength of a firm's dynamic capabilities determines the speed and degree (and
associated cost) of aligning the firm's resources—including its business model(s)—with
customer needs and aspirations. To achieve this, organizations must be able to
continuously sense and seize opportunities, and to periodically transform aspects of the
organization and culture so as to be able to proactively reposition to address yet newer
threats and opportunities as they arise. Terse 2018

Dynamic capabilities are hard for rivals to replicate because they are built on the
idiosyncratic characteristics of entrepreneurial managers and the history-honed routines
and culture of the organization (Teece, 2014a). In addition, there is the uncertain
imitability of a complex system that even those directly involved may not fully understand
(Lippman and Rumelt, 1982). Because they are a unique and valuable general-purpose
resource, strong dynamic capabilities can serve as a firm foundation for sustainable
competitive advantage. This is especially true the more deeply embedded the capabilities
are in the organization, and the less they are resident only in the top management
team. Teece 2018

 Managers must articulate a vision and establish an appropriate organizational culture and an incentive system that will promote organizational identification and loyalty (Augier and Teece, 2009).
 
 

Put differ- ently, selecting the right ‘architecture’ for a busi- ness requires not just understanding the choices available; it also requires assembling the evidence needed to validate conjectures and hunches about costs, customers, competitors, complementors, dis- tributors, and suppliers. Designing

Distinct rolle

H1: Cognitive frames mediates the effect of dynamic capabilities on competetive advantage
above and beyond mediating effect of operational capabilities

H2: Cognitive frames has a direct impact on operational capabilities

H3: Something to do with dynamism - importance of cognitive frames increases in dynamism

Minor hypotheses: On 

The role of cognitive frames in strategy: Schilling 2018 abstraction


Proactiveness
Coff and Blyler - who reaps the reward from rents). cognitive frames common -
proactiveness - culture can prohibit the rent appropriation happening 

Cognitive programming as rent appropriation inihibitor

Cognitive programming as implementation catalyst (complement to OC)

Cognitive programming as alignment mechanism (link to OC)






Ethnographic study on cognitive farmes - still top management team
http://knowledge.wharton.upenn.edu/wp-content/uploads/2013/09/13481.pdf

Implications: Maybe proactiveness cog frames can help in the paradox schreyogg, and the
elephant. 

We suggest that a systematic socio-cognitive approach to “competitive sensemaking” has
been absent from theory and research on competitive strategy. We define competitive
sensemaking as the social and cognitive processes that underlie how firms detect, define,
and conceptualize their competitive relationships with other firms. Competitive
sensemaking is a subset of the more general process of strategic sensemaking, which is the
“making plausible sense” of the broad array of stimuli and circumstances that characterize
complex market situations. Using the value-based view of value creation and capture as a
conceptual base for our arguments, we unpack four cognitive underpinnings of competitive
sensemaking: mental time travel, comparability, counterfactual reasoning, and stories. We
then show how these four components were differentially involved in shaping competitive
sensemaking in four actual market situations. In doing so, we illustrate how competitive
sensemaking provides fundamental inputs into the value creation and value capture
process. We conclude the paper by drawing out the implications of competitive sensemaking
for strategy theory and research.



Some strategic decisions will long be remembered for being remarkably successful, providing dramatic benefits to customers, investors, or social welfare. The individuals who make these decisions are often called "visionary." But what enables some people to be visionary? Is it a gift or skill, and can it be learned? In this paper I identify three main cognitive processes that underlie the creation of visionary strategy: abstraction, idealism, and long paths of analytical reasoning. Using a combination of examples and cognitive science, I explain how they work individually and in combination, and how they may be cultivated.
Schiller




Cognition explain organizational inertia garud and rappa
Sharing of dominant logic pralahad and bettis (actually managmenet but maybe org as well)
Tripsas and Gavetti 2000 focus on management cognition in digital imaging

power of analogical reasoning prevailing in teh strategy makin in novel and complex worlds

(Gavetti, 2012; Gavetti et al., 2005; Tripsas and Gavetti, 2000), awaiting from LB


DCs are viewed as stable patterns of action that poten- tially build a string of temporary
advantages by adding, integrating, and reconfi guring resources, which amount to sustained
advantage once the entire pattern is taken into account (Blyler and Coff, 2003)

Again, some acquiring fi rms (not many) undergo these motivational change processes (Gottschalg and Zollo, 2007) by

would for the larger part argue that 
including the 33000 people at nokia / cultural change rr



Capabilities changes reinvent productivity and business processs, as well as introcuing
new capabiltieis. 
Cloud based capabilities and infrastructure


'AS I conlcluded my talk that morning in orlando, i focused on what would be our grandest
endeavor / the biggest hurdle, transforming the microsofg culture. p 89

Defined culture as a 'complex system of individual mindsets' culture is how an
organization thinks and acts, but indiciduals shape it. p 90

%https://news.microsoft.com/2014/02/04/satya-nadella-email-to-employees-on-first-day-as-ceo/
%https://qz.com/work/1539071/how-microsoft-ceo-satya-nadella-rebuilt-the-company-culture/

%{\bf How is a direct link between dynamic capabiliites and competetive advantage working
%  and under which conditions does it come into play or not?}

In the face of a growing consensus on the definition (ref) and outcome (ref) of dynamic
capabilities as working through mediators, the presence of direct effects argued by others
(ref) seems puzzling. However, the broader empirical literature on the antecedents of firm
performance offers an interesting point of departure. This literature concerns itself with
the relative importance of factors at the firm-level on the one hand, and those at the
industry level at the other. 

The debate over the relative role of different levels of factors in accounting for the
major share of the variance in firms performance has been of particular importance in the
field of strategic management, most notably in the Strategic Management Journal
(M. A. Fitza 2014; Hansen and Wernerfelt 1989; McGahan and Porter 1997; Rumelt
1991). Overall, the accumulated empirical evidence show that firm factors matter far more
for firm performance than industry factors (McGahan and Porter 2002). These stylized facts
have been used as empirical support for the importance of idiosyncratic firm resource and
capabilities; leading up too recent formulation of the resource and capability based
theorizing (e.g. (Barney 1991; Teece, Pisano, and Shuen 1997)). However, the dominance of
the firm-level explanations have been partly refuted by later scholars finding the sum of
indirect and direct effects from industry to be of more importance than previously
indicated by (Mcgahan etc). Moreover, the industry effect is also more time invariant
(i.e. more stable) compared to firm level effects. Recent findings also suggest that there
are considerable interaction effects between industry and the firm indicating that
industry membership works directly at performance as well as indirectly through affecting
the firm. However, little or no explanations are made for the mechanisms of this
interaction.In other words, industry factors matter alas less than firm-level factors, and
the indirect link from industry-level through firms are in need of further development.

One way of looking at this is to consider the dynamic capabilities of the firm. Although
an increasing consensus focuses on these dynamic capabilities as the ability to create,
extend and modify firm resources in the face of change, there still looms an important
stream of literature suggesting that dynamic capabilities can materialize as management
actions (ref eisenhardt, peteraf, helfat). A management action aimed at strategic change
will very often entail "resource orchestration" as broadly discussed in the literature
(Helfat etc). But, strategic change are also found to come about from adapting and
changing the positioning of the firm within its industry (ref) or simply refocusing its
focal industry (ref). In other words, when a firm with excellent dynamic capabiliites
senses and seizes opportunities within its domain, the implementation of strategic change
will often both entail a considerable share of "asset orchestration" (helfat etc) as well
as management actions directed at positioning within its industry (ref). One way to
illustrate this is to consider Microsofts strategic change from operating system to cloud
provider. CEO Satya Nadela and his team was able to restart the business by creating,
extending and modifying the resources of the firm, including new skills, routiens and
hardware, as a response to a growing demand for cloud computing (ref nadellas
book). Meanwhile, his vision and formulated strategy also direclty put the company into a
new industry, or at the very least reshaped the position they were already in. While
Microsofts dynamic capabiliites worked significantly through "resource orchestration",
the sensing and seizing of the management also ordered a direct re-positioning at the same
time.  

Moreover, The link between firm resources and competitive advantage is determined by the
extent to which resources are valuable and difficult to imitate \citep{Barney1991a}. This
principle is also used to evaluate how higher order capabilities (e.g. dynamic
capabilities \citep{Winter2003,Collis1994,Teece2007}. Hence, for a direct link to exist
the dynamic capabilities of the firm will have to be valuable and difficult to
imitate. This is an area of certain contention in the literature \citep{Peteraf2013} which
we get back to below, but an increasing majority is arguing that dynamic capabilities
reflect more complex processes that co-evolve with responses to market changes over time
and hence creating idiosyncratic capabilities that are unique, valuable and hard to copy
\citep{Helfat2007,Arndt2018}. Consequently, they are likely to lead to competitive
advantage.

In most cases such direct effects between dynamic capabilities and strategic change
resulting in competitive advantage, are combined with the indirect effects (i.e. resource
orchestration). To test this proposition empirically is hard, but one hypothesis following
from this is related to the role of mediators. if a direct path does not exist, the
relationship between dynamic capabilities and competitive advantage would be fully
mediated by the proper set of firm resources. 

\emph{H1: The existence of a direct path from dynamic capabilities to competitive advantage
suggest that full mediation is not present}
 

Even if the dynamic capabilities of the firm themselves are not valuable and inimitable as
needed to create competitive advantage, the underlying resources of the firm may very well
be. Such resources in the form of operating routines or capabilities can in turn stem from
an inherent learning process evolutionary distinct to the firm. Operating routines can
thus lead to competitive advantage because of their evolutionary character
\citep{Nelson1982,Winter2003}. To the extent operating routines stems from idiosyncratic
evolutionary process of individual firms rather than being the result of best practices,
they can satisfy the VRIO condition and thus be a source of competitive advantage.  Hence,
through "resource orchestration" a firm with non-valuable dynamic capabilities can achieve
competitive advantage through the indirect path. The flip side of this is that a valuable
dynamic capability would not be able to create competitive advantage by orchestrating
non-valuable resources. Consequently, the mediation effects of underlying resources is
dependent on the nature of the resource - it being valuable and inimitable or not.

\emph{H2a: When dynamic capabilities are being mediated by VRIO resources the firm can
  achieve competitive advantage}

\emph{H2a: When dynamic capabilities are being mediated by VRIO resources the firm can not
  achieve competitive advantage}

DEspite a growing consensus on the nature of dynamic capabilites, a serious contention is
presen tin the literature \citep{Peteraf2013}. This stems from different notions of
dynamic capabilities in the two "founding" papers of the theory, \cite{Eisenhardt2000} and
\cite{Teece1997}.

One strand of contributions sees DC as routines that are simple in nature and more as best
practices to consider \citep{Eisenhardt2000}. Thus, dynamic capabilities are more like
behavioral patterns and less idiosyncratic to the firm and will consequently not be able
to satisfy the VRIO condition for competitive advantage to emerge. Dynamic capabilities as
such routines are short lived, unstable and “substitutable” (ibid, p 1110) thus violating
a key VRIO condition, and they are more “more homogeneous (..) than is usually assumed”
(ibid, 1116). However, the simplicity of simple rules also makes them faster do deploy and
put in motion, albeit potentially unstable. To the extent dynamic capabilities manifest
themselves as simple rules they cannot alone create competitive advantage, but could very
well generate strategic change by facilitating the process of resource orchestration.. In
other words, even if dynamic capabilities play out as simple rules that are not in
themselves satisfying the VRIO condition, the way they work through creating, extending
and modifying firm resources, can still yield competitive advantage through orchestrating
VRIO resources.

On the other hand, \cite{Teece1997} and others \cite{Helfat2007} argue that dynamic
capabilities will rather invariably be VRIO due to the evolutionary nature of their
development \citep{Arndt2018,Peteraf2013}. Arguing that dynamic capabilities reflect more
complex processes that co-evolve with responses to market changes over time and hence
creating idiosyncratic capabilities that are unique, valuable and hard to copy
\citep{Helfat2007,Arndt2018}. Consequently, they are likely to satisfy the VRIO condition
and thus leading to competitive advantage.

The contention between these streams of literature is most explicit in highly dynamic
environments (distefano, peteraf) suggesting that environmental dynamism could play into
the presence of direct and indirect linkages between dynamic capabilities and competitive
advantage. This is also supported in previous empirical literature \citep{Schilke2014}.
This could suggest that dynamic capabilities exhibit different characteristics under
various levels of environmental dynamism. This is also suggested by Eisenhardt blaboblabla
arguing that under moderatly dynamism dynamic capabilities takes the form of best
pracitces (e.g. aquisition targeting) and is less important than under more dynamic
environments. However, albeit best practices, the way they are implemented in the factor
markets are still idiosyncratic. And taken into conjunction with the routinzed process of
changing underlying routines, VRIO will very likely emerge under more stable
conditions. It is not that the market/direct expression of dynamic capabilities are not
present in stable environments, rather it is less pronounced and the indirect mediated
effect (evolutionary path) is more important. In other words, it is not neither nor, but
degrees of importance.

\emph{H3: The mediation effects of underlying resources is dependent on the environmental
dynamism facing the firm }







%% Increasingly the literature remain focused on an \emph{indirect
%%   path} between dynamic capabilities and competitive advantage
%% \citep{Barreto2010,Schilke2018}. The theory behind this path goes back to the core of
%% dynamic capabilities where resource orchestration happens as a response to changes in the
%% environment. When markets, technologies or other external factors changes, firms with high
%% levels of dynamic capabilities are able to seize opportunities for changes, seize these
%% opportunities and transform the resource base to fit the strategic change needed to stay
%% competitive \cite{Teece2007}. This ability to adapt is by some coined \emph{evolutionary
%%   fitness} defined as BLABLBALBA and stems from the ability of the firm to reconfigure
%% resources in the face of change. The extant literature, however, is less clear on how the
%% nature of the underlying resources matter in this indirect path. On the one hand we can
%% see 

%% Following REF suggesting more reseaarch on dynamic capabilities should put forth
%% competeing hypotheses we propose these first 

%% We introduce the notion of multilevel VRIO to integrate the interactions between different
%% types of capabilities at different levels. This entails that dynamic capabilities (the
%% high level) can consist of complex and evolutionary characteristics that are idiosyncratic
%% to the firm and hence VRIO by definition along the lines of how the concept is understood
%% by \cite{Teece1997}. On the other hand dynamic capabilities can also take the shape of
%% simpler routines which are short lived, unstable and substitutable thus violating a key
%% VRIO condition. This follows the notion of dynamic capabilities found in
%% h\cite{Eisenhardt2000} arguing that they are more “more homogeneous (..) than is usually
%% assumed” (EM: 1116). 

%% Moreover, these dynamic capabilities are in the extant literature argued and found to work
%% directly \citep{Kor2005,Eriksson2014,Schilke2018} on competitive advantage which suggests
%% that they are indeed VRIO. On the other hand, a growing body of literature (ref) as well
%% a consensus on the definition of dynamic capabilities (ref), holds that they can lead to
%% competitive advantage by changing underlying resources and routines. In this
%% \emph{indirect path} the dynamic capabilities itself need not be VRIO as long as the
%% underlying resources themselves are. 

%% \subsection{The role of dynamism}

%% The most contentious issue in the current debate on dynamic capabilities is how their
%% nature varies with the environment. 

%% If htey are indeed latent 


%% The lower level resources and routines are to be understood as conventional resources in
%% line with the resources-based view 


%% WHAT IS THE ROLE OF DIRECT VS INDIRECT, HOW DOES IT RELATE TO THE ENVIRONMENT, AND WHAT
%% ROLE DOES THE TYPE OF UNDERLYING RESOURCE PLAY



%% Conventional dc literature has argued for both indirect and direct links. Indeed both
%% have been found, but while the indirect link is well rooted in the theory, the direct link
%% is to our knowledge less clear. The core idea of DC is that it works to modify underlying
%% resources as a response to changes in the environment and hence creating strategic change
%% needed to stay competitive. This resource orchestration is heavily represented in the
%% literature (ref) and through this process firms are able to generate CA. In such instances
%% the CA can stem from VRIO resources and/or VRIO DC. A VRIO DC would mean that the firm is able
%% to shuffle and reorganize generic resources in a way that creates a new and unique
%% configuration of resoures idiosyncratic and valuable to the firm. In such instances, the
%% outcome whould be CA. On the other hand, the DC may not by itself be VRIO but rather a set
%% of simple routines put in motion to change, but if the resources they reconfigure are
%% themselves VRIO, the outcome could just as eqsily be a competetive advantage. this
%% equfinaity is discussed in the literaure \cite{DiStefano2010} and makes it hard to
%% distingusih between the role of DC and ordinary capabilites in generating CA. 

%% Furtemore, Considering a direct link between DC and CA, DC it self must consist of some
%% VRIO characteristics to garner CA. So in owther words, when is DC working directly and
%% indirectly_ 

%% How are dynamic capabilities working directly on competitive advantage? The microfoundations of dynamic capabilities explicated by \cite{Teece2007} is a useful
%% point of departure of this theoretical inquiry. He hold that for "analytical purposes,
%% dynamic capabilities can be disaggregated into the capacity (1) to sense and shape
%% opportunities and threats, (2) to seize opportunities, and (3) to maintain competitiveness
%% through enhancing, combining, protecting, and, when necessary, reconfiguring the business
%% enterprise’s intangible and tangible assets". Sensing and seizing is a part of what we
%% would coin the "opportunity capture" of dynamic capabiliites, where as reconfiguring
%% concerns itself with the "resource orchestration" needed to change assets to stay
%% competetive. 

%% Hence, when considering indirect and direct paths to compettetive advantage we will have
%% to be clear on two fronts: Are DC themselves VRIO and are the underlying resources and
%% routines central to the resource orchestration, VRIO. 


%% > 

%% The microfoundations of dynamic capabilities—the distinct skills, processes, procedures, orga- nizational structures, decision rules, and disciplines—which undergird enterprise-level sensing, seizing, and reconfiguring capacities are difficult to develop and deploy.
%% For analytical purposes, dynamic capabilities can be disaggregated into the capacity (1) to sense and shape opportunities and threats, (2) to seize opportunities, and (3) to maintain competitiveness through enhancing, combining, protecting, and, when necessary, reconfiguring the business enter- prise’s intangible and tangible assets.

OC som mindset / non-vrio? organizational mindset
DC influcences IOC 
DC influences VOC


two paths / changes resources vs change mindset

why so interested in operating routines resources, what about other mechanisms
Can dc change collective cognition? 

ENTP - EO coen and slevin

%%        name Mean_est     t
%% 3   DC -> P    0.396 6.157
%% 4   DC -> R    0.416 8.104
%% 5 DC -> CA1    0.239 4.018
%% 8  P -> CA1    0.172 2.724
%% 9  R -> CA1    0.081 1.262
%% > R
%%        Path   LOW     t  HIGH     t
%% 1 DC -> CA1 0.417 6.585 0.144 1.513
%% 2   DC -> R 0.388 6.078 0.534 7.525
%% 3  R -> CA1 0.017 0.219 0.211 2.123
%% > P
%%        Path   LOW     t  HIGH     t
%% 1 DC -> CA1 0.355 4.805 0.133 1.379
%% 2   DC -> P 0.386 5.294 0.436 4.663
%% 3  P -> CA1 0.155 2.193 0.232 2.315
%% > B
%%        name R_LOW_est   t.x R_HIGH_est   t.y
%% 1 DC -> CA1     0.355 5.407      0.044 0.400
%% 2   DC -> P     0.376 3.729      0.432 4.668
%% 3   DC -> R     0.366 5.569      0.522 7.225
%% 4  P -> CA1     0.158 1.989      0.220 2.357
%% 5  R -> CA1     0.010 0.135      0.200 2.033
%% > 

%% %% creating and One example
%% %% is the role of post-acquisition routines where routines to identify and deal with
%% %% acquisition partners are a source of generating strategic change (Eisenhardt and Sull
%% %% 2001, ref). Another example is (see refs from Schilke 2018). The functioning of this
%% %% expression of dynamic capabilities are more latent and hence mainly observable ex
%% %% post. Hence we can see this as direct route to competitive advantage as this expression of
%% %% dynamic capabilities work directly in the factor market in capturing resources
%% %% needed for strategic change. We coin this direct connection as the \emph{market path}.

%% On the other hand, a stream of contributions see dynamic capabilities as evolutionary and
%% more complex routines (ref) or processes (ref) set to change underlying capabilities in
%% the face of changing conditions (TPS). This notion stems from evolutionary economics
%% \citep{Winter2003,Nelson1982} and the behavioral theory of the firm \citep{Cyert1963}
%% holding that path dependency and complex processes for changing underlying operating
%% routines are a source of competitive advantage for the firm. Along this indirect path
%% change is gradual and running through changes in underlying routines. We coin this the
%% \emph{evolutionary path}.

%% These paths from dynamic capabilities to competitive advantage are different expressions
%% of dynamic capabilities with different mediums of intermediation. The \emph{market path}
%% works in the market and his hence not typically altering lower order capabilities and
%% resources in the firm directly, but rather acquiring new ones. The source of competitive
%% advantage along this path stems from the interaction with the other, \emph{evolutionary
%%   path} in a dynamic bundle \citep{Peteraf2013}. Such interactions create a bundle of
%% capabilities and routines that are socially complex and hard to imitate
%% \citep[p. 320]{DiStefano2014}. Moreover, the interactions between the two paths of dynamic
%% capabilities is separable with respect to what intermediation exist. Consequently,
%% separate paths exist albeit a part of a dynamic bundle.

%% \cite{DiStefano2014} propose an integrated framework where simple routines and more
%% complex processes could contribute simultanously to the firms outcome, specifically its
%% competitive advantage. In this framework, coined the "organizational drivetrain" the
%% authors envision the movement of an organization as set in motion by a drivetrain system
%% not unlike that of a bicycle.

%% Building on the notion of an organizational drivetrain we argue that dynamic capabilities
%% can indeed impact the organizational movement or change through two distinct but
%% interconnected mechanisms (i.e. the crank set and the rear shifter). The two mechanisms
%% consitutting the drivetrain is consistent with our notion of \emph{evolutionary path} and the \emph{market path}. The
%% evolutionary path is instigated by more complex processes setting in motiong a chance in
%% the lower order (operational) routines \cite{Collis1994,Winter2003}. In the drivetrain
%% metaphor this path corresponds to change stemming from the front crank set of thich there
%% are fewer gears, but with larger impact. This path is slower
%% and yields a more idiosyncratic set of organizational routines which is very much in line
%% with the idea put forth in the seminal founding papers \citep{Teece1997,Winter2003}.

%% Seeing the bike rider and her efforts to handle the levers of control (i.e. the gear
%% shifter) as the dynamic capability, and consider it not only as a management effort, but
%% that of the organization as a whole. The process of changing front gears are more profound
%% (i.e. evolutionary) and costly, and it is for larger and slower strategic
%% changes. Operating with simple routines in factor markets, however, is quicker and the
%% rider does this on a more frequent basis with less effort.

%% H1: Dynamic capabilities works to create competitive advantage through two distinctly
%% different paths.

%% \subsection{The two paths of dynamic capabilities}
%% The objective of strategic change is to enhance a firm’s competitive advantage
%% (ref). Doing so over time makes for the evolutionary fitness of the firm
%% \citep{Helfat2007}. When firms are developing capabilities for strategic change through
%% creating, extending and modifying the resource base this means making decisions to develop
%% operational routines. Such routines can in turn stem from an inherent learning process
%% evolutionary distinct to the firm. Operating routines can thus lead to competitive
%% advantage because of their evolutionary character \citep{Nelson1982,Winter2003}. To the
%% extent operating routines stems from idiosyncratic evolutionary process of individual
%% firms rather than being the result of investments in factor markets, they can satisfy the
%% VRIO condition and thus be a source of competitive advantage. In such instances, dynamic
%% capabilities work through operating routines in creating competitive advantage. This
%% indirect impact on competitive advantage is very much at the core of the DC literature
%% (ref). Firms with high dynamic capabilities will be able to better change operating
%% routines to facilitate strategic change in response to changing conditions facing the
%% firm. Dynamic capabilities then creates VRIO operating routines and hence competitive
%% advantage. When this happens, firms changes their operating routines in a evolutionary
%% making operating routines themselves OR itself satisfies the VRIO condition.

%% \emph{H2: Dynamic capabilities has an indirect effect on competitive advantage through
%%   changing the operating routines of the firm.}

%% On the other hand, resources and capabilities can to a certain extent be captured in
%% factor markets through investment decisions (ref) or through alliances (ref) or M\&A
%% activates (ref). By acquiring resources through investments, the operating routines of the
%% firms may not be directly affected. When these acquisitions happen as a response to
%% systematic resource orchestration (ref), i.e. reshaping the resource mix to obtain
%% coordinated resource deployments \citep{Kor2005,Pan2006}, they would indeed have potential
%% performance implications for the firm. Moreover, when this process of resource acquisition
%% process of DC develops a way to acquire and mix resource for the purpose of adapting to
%% change, the resources themselves would not necessarily satisfy the VRIO condition. The way
%% in which this happens may be much harder to imitate due to the complexities such routines
%% exhibit. Although resources themselves exhibit clear equifinality in their impact on
%% competitive advantage, combined with the simple routines of the orchestration
%% (i.e. dynamic capabilities), the sum would likely satisfy the VRIO condition. This is
%% especially true if routines are deployed rapidly in changing market conditions. In other
%% words, there are several ways in which resources can be orchestrated to generate CA,
%% whereas improved OR will always mean improved competitiveness et ceteris paribus. In such
%% instances DC itself can lead to CA directly.

%% \emph{H3: Dynamic capabilities has a direct effect on competitive advantage through the
%%   process of resource orchestration and acquisition of resources in factor markets.}

%% In other words, we would expect DC to have both a direct and an indirect impact on CA. The
%% indirect impact stems from the \emph{evolutionary path} of complex processes, whereas the
%% direct impact stems from the \emph{market path}.  Although both the direct and the
%% indirect path stems from DC creating, extending and modifying the resource base (OR or
%% factor market resources), the difference in the path’s stems from to what extent the
%% mediating construct alone can by itself satisfy the VRIO condition. Operating routines
%% can, due to their evolutionary nature, in turn shaped by DC, achieve CA
%% \citep{Collis1994,Winter2003}. Factor market resources and market acquired capabilities
%% and best practices can not \citep{Eisenhardt2000,Peteraf2013}, but in conjuncture with the
%% routines of acting in factor markets as well as the speed of deployment, the sum of
%% dynamic capabilities and factor market action would lead to competitive advantage.

%% However, the interlinkages between simple routines (the market path) and the coplex
%% processes (the evolutionary path) would vary with the environment.

%% \subsection{Interplay under environmental dynamism}

%% \cite{DiStefano2014} note that "even if specific simple rules are unstable and ephemeral,
%% the system as a whole is not" and that the "real source of sustainable competitive ad- vantage for an enterprise is the difficulty of imitating and substituting for the entire dynamic bundle that the system represents" (p.sss).

%% But the interlinkages between the parts of the system (the evolutionary- and the market
%% path), are likely to vary with the environmental dynamism. Under moderatly dynamism
%% dynamic capabilities takes the form of best pracitces (e.g. aquisition targeting) and is
%% less important than under more dynamic environments. However, albeit best practices, the
%% way they are implemented in the factor markets are still idiosyncratic. And taken into
%% conjunction with the routinzed process of changing underlying routines, VRIO will very
%% likely emerge under more stable conditions. It is not that the market/direct expression
%% of dynamic capabilities are not present in stable environments, rather it is less pronounced and the indirect
%% mediated effect (evolutionary path) is more important. In other words, it is not neither
%% nor, but degrees of importance.

%% Moreover, in high velocity environments the role of the evolutionary path tends to
%% weaken as such path-dependent changes occur slower compared to the market path. From the
%% market path with its simple routines we can argue that creation of "new knowledge” to
%% “allow for emergent adaptation” (EM, p. 1116) is the consequence. Consequently, the market
%% path is stronger and more prevalent under higher levels of environmental dynamism.

%% \emph{H4: Dynamic capabilities works relatively stronger through the market path in highly
%% dynamic environments.}


To see how dynamic capabilities work as 'coach' we consider two 
Looking at dynamic capabilities through the functions of sensing and seizing opportunities
\cite{Teece2007} makes an obvious case for resource orchestration which is also
highlighted in the extant literature. This function is related to the discovering and
capturing opportunities (e.g. the ability of the organization to sense and seize
opportunities (ibid)) which is very much related to the concept of exploration
\citep{March1991}. Moreover, as discussed above, these functions are also requiring a
certain non-behavioral object of action in order to be fully efficient. To achieve a
competitive edge firms will have to be able to sense and seize, which require a certain
organizational cognitive frame to be efficient. 

changes resulting from execution of dynamic capabilities (e.g. sensing and seizing) may
move existing organizational objective functions and thus render current incentive
alignment obsolete or out of place. This leads to an objective displacement that is
particularly prevalent in firms with high sensing and seizing capabilities because distant
search may lead to more radical change (ref). Organizations with high levels of dynamic
capabilities geared towards sensing and seizing opportunities are doing so through
\emph{resource orchestration}, e.g. by creating, extending and modifying the resource
base, as discussed extensively in the extant literature
\citep{Helfat2007,DiStefano2014,Protogerou2012,Schilke2018}. However, sensing and seizing
opportunities in the market relates to discovering market disequilibrium
\citep{Kirzner1997} as well as upsetting existing equilibria \citep{Schumpeter1934}. The
process of sensing and seizing 'involves understanding latent demand, the structural
evolution of industries and markets, and likely supplier and competitor responses'
\citep[p. 1322]{Teece2007} as well as 'a good deal of customer, competitor, and supplier
information and intelligence (with a) a significant tacit component' (ibid p 1330).To make
sure the organization respond to changes (i.e. having an efficient cognitive frame)
stemming from executed dynamic capabilities (e.g. sensing and seizing) the firm needs to
make sure incentives are aligned for change and that the organization buys into the changes.

First, the \emph{incentive alignment problem} relates to how interests of the firm
as a whole and the collection of individual are aligned \cite{Gottschalg2007} and how this
alignment changes when the firm embarks on strategic change as triggered by execution of
dynamic capabilities. Through the process of exploration, individuals in the organization
may accumulate private and tacit information that, if unchecked, can become a source of
knowledge asymmetries and incentive problems (alchian and demsetz 1972), and leave the
firm vulnerable to agency problems. Accumulation of private information in the
organization without incentive alignment of some sort can lead rent-appropriation from
information asymmetries \citep{Blyler2003}. In such instances of potential misalignment of
interest, a shared cognitive frame emphasizing a growth mindset can be a useful remedy.

Second, the \emph{buy-in problem} refers to the general willingness of the organization to
accept and value the importance of some decision or system 'without being prompted or
required by formal governance mechanisms'\citep[p. 812]{Simsek2009}. When individuals of
a firm buys into changes they will enhance the full organization`s ability to not only act
upon changes (Marchand et.al 2000), but also in 'timely responsiveness' (Kohli et.al 1993,
Simsek2009). 

Both the incentive alignment problem and the buy-in problem are interrelated, but still
represents different mechanisms. They are tied to individual motivation albeit in
different ways. The incentive alignment problem is tied to \emph{extrinsic motivation}
(Lindenberg 2001, Brief and Aldag 1977) where the individual is motivated by rewards or
outcome 'such as money, power, recognition and so forth'
\citep[p. 420]{Gottschalg2007}. The buy-in problem, in turn, is related to \emph{intrinsic
  motivation} where the individual is driven by the goal of being engaged in enjoyable
(Lindenberg, 2001), self-determined, and competence-enhancing (Deci \& Ryan, 1985)
behavior (hedonistic intrinsic motivation), as well as adhering to a goal of engaging in
behavior that is compli- ant with norms and values (normative intrinsic
motivation). \todo{some direct citations here - need rewrite}.

Extrinsic motivation is often addressed through reward systems (ref) but tacit mechanisms
such as psychological ownership \citep{Sieger2013} has been found to be very efficient in
aligning incentives. Psychological ownership basically retains agency theory’s assumption
of the self-interested manager, whereas the non-economic functions of ownership, such as
efficacy, identity, and territoriality, might curtail expropriating behaviour and align
the interests of agents and principals (cf. Pierce et al., 2003). Hence, this concept of
psychological ownership bridges over to the intrinsic motivation of individuals. Moreover,
intrinsic motivation in general is influenced through changing the tasks of individuals or
at the very least accepting individuals to change ttasks themselves to fit their
perception of their context (Hackman and Gersick), and through regimes of socialization
and shared norms (Kerr and Jackofsky). Common for both psychological ownership and the
intrinsic motivation enhancement in general is a feature of the employees' cognitive
framing, which guides the mental pathways by which they decide what is, and what is not,
important. According to Leonard- Barton (1992), these norms and values represent the
accretion of decisions made over time in response to the interpretation of organizational
roles. Consequently, a cognitive frame that enhances the intrinsic motivation of the
employees are very valuable. One way of thinking of instilling an appropriate and
efficient cognitive frame would be the concept of a growth mindset (Dweck) as also
emphasized in the case of Microsoft.    

To see how this would work recent contribution in neuroscience is helpful. Several
contributions have established a clear connection between having a growth mindset (Dweck 200x) and
emphasizing intrinsic motivation (Schroder et.al 2014,O`Rourke et.al 2014). Moreover,
individuals with higher degrees of intrinsic motivation from a growth mindset exhibit a
significant higher ability for learning and consequently for embracing new learning
opportunities (Yeager et al 2012). In other words, individuals with a growth mindset are
more open to change because it enhances their intrinsic motivation from learning which
fuels their individual utility (Ng 2018). In a situation with considerable change stemming
from execution of dynamic capabilities, an organization with a common cognitive frame
based on a growth mindset will be able to capture more value from the change \emph{et
  ceteris paribus}.

Consequently, by instilling a growth mindset as a change in the cognitive frame of
Microsoft, Nadella enhanced the change ability of the firm by strengthening the intrinsic
motivation employees have to buy into the change, and meanwhile securing a certain
interest alignment. So Nadella did not only change the resources and capabilites of the
company due to a strategic change towards the cloud, he did also instill a growth mindset
to utilize the motivational and non-behavioral object of action discussed above. In other
words, dynamic capabilites can work through two distinctly different paths. The first path
is the familiar process of changing underlying resources and operational capabilities
(\emph{resource orchestration path}). The second is the focal path of this paper where
dynamic capabilities changes the cognitive frames of the organization \emph{cognitive
  programmer path}. This leads us to the following initial hypotheses:

H1: Dynamic capabilities influences the quality? of operational capabilities 

H2: Dynamic capabilities influences the WHAT of cognitive frames 

The link between operational capabilities enacted through dynamic capabilites, and
competetive advantage is well established in the literature (refs). The main notion is
that resources and capabilites evolved through dynamic capabilites are inimatable and
rare, and hence conducive to the conditions underpinning competetive advantage (refs
barney). Thus our third hypothesis is well known as the first one:

H3: Operational capabilites acts as a mediator between dynamic capabiliites and
competetive advantage. 

Similarly, instilling a growth mindset is also leading to competetive advantge BECAUSE???
\todo{Need to discuss this}


And one more hpythoses on how COG influences resources



THIS IS HOW FAR I CAME

the general capability, views, norms and rules of behavior, with regard to accessing,
understanding and using information in a social collectivity. Zheng 2005

To make sure that these interests are aligned



DC executed requires changes. These changes will require a simultanous change in cognitive
frames of the firm for N reasons.

1. Sensing and seizing activities typicaly of dynamic capabilities generate possible
agency problems *blyler coff
2. Information from the process is interpreted and differently at different levels in the
corporate hierarchy (gavetti) hence aligning interpretation across the organization matter

One way of overcoming this agency problems and varying interpratations of the environment,
is to develop a common cognitive frame within which all information is interpreted and
acted upon. Such common traits purpose and organizational identity, 
organizatioanl values and culture \citep{Verona2011}.



To see how dynamic capabilities acts through cognitive objects of action (i.e. as a
cognitive programmer) we consider two distinctly different functions of dynamic
capabilities related to their microfoundations \citep{Teece2007}. The first function is
related to the discovering and capturing opportunities (e.g. the ability of the
organization to sense and seize opportunities (ibid)) which is very much related to the
concept of exploration \citep{March1991}. Organizations with high levels of dynamic
capabilities geared towards sensing and seizing opportunities are doing so through
\emph{resource orchestration}, e.g. by creating, extending and modifying the resource
base, as discussed extensively in the extant literature
\citep{Helfat2007,DiStefano2014,Protogerou2012,Schilke2018}. However, sensing and seizing
opportunities in the market relates to discovering market disequilibrium
\citep{Kirzner1997} as well as upsetting existing equilibria \citep{Schumpeter1934}. The
process of sensing and seizing 'involves understanding latent demand, the structural
evolution of industries and markets, and likely supplier and competitor responses'
\citep[p. 1322]{Teece2007} as well as 'a good deal of customer, competitor, and supplier
information and intelligence (with a) a significant tacit component' (ibid p
1330). Consequently, in this process of exploration individuals in the organization may
accumulate private and tacit information that, if unchecked, can become a source of
knowledge asymmetries and incentive problems (alchian and demsetz 1972), and leave the
firm vulnerable to agency problems. Accumulation of private information in the
organization without incentive alignment of some sort can lead rent-appropriation from
information asymmetries \citep{Blyler2003}. One way to  if not sufficiently aligned with the
cognitive frames of the organization. This particular





The other path, the market path, is characerized by smaller routines and practices that
themselves are not satisfying the VRIO condition (preciclsy as argued by EM). However, in
factor markets  
simple routines and complex processes can lead to comp


Market path: simple rules guiding actions in markets for aquistion of assets. Not
themselves VRIO, but taken together with the assets they aquire more idiosyncratic to the
firm. These simple rules are routines and they corresponds to rapid changes and the rear
wheel. mor profound in dynamic environments
Eisenhardt and Sull (2001) credited its prowess in this arena to the simple rules that Cisco has developed to guide its acquisition process. Cisco

Evolutionary path: More complex rules and process working to change the operational
routines of the firm. These are themselves idiosyncratic and VRIO (vrio stems from the DC
level and hence generates path dependent routines at theOR level. Corresponds to more
gradual change. the front crank / more gradual environments. More complex processes need
to work thorugh a mediation (OR). More simple rules works more directly (unobserved)

\cite{DiStefano2014} propose a metaphore to think of these paths to competetive
advantage through the idea of the "organizational drivetrain". 

The interplay prposed by DiStefano 

We find that DC as routines to shape routines is most prevelant under moderate dynamism,
whereas DC as routines to capture resources in factor markets are more prevelant under
dynamic environments. 

Under certain conditions the DC operates through changing underlying routines. OTher times
they work through factor markets such as EM and the Cisco case. 

Routinization indicates the extent to which organizational pro- cesses are stable and repetitive (Nelson &Winter, 1982) and constitutes an important revelatory access to the nature of dynamic capabilities (Barney & Felin, 2013). The sub-stream around Teece et al.’s (1997) seminal work largely argues that dynamic capabilities rely on highly rou- tinized processes, whereas the sub-stream around Eisenhardt and Martin's (2000) conceptualization argues for reduced routinization (Peteraf et al., 2013; Schreyögg & Kliesch-Eberl, 2007)


Start with two contradictory definitions:
Kale and Singh (2007, p. 982) ex- plained that “dynamic capability refers to the capac- ity of an organization to purposefully create, extend, or modify its resources or skills.”

Zollo and Winter (2002, p. 340) explained that “a dynamic capability is a learned and stable pattern of collective activity through which organizations systematically generate and modify operating rou- tines for improved effectiveness.”



One contentious issue surrounding the DC theory is the extent to which differing
views on the nature of dynamic capabilities \citep{Peteraf2013}. One strand of
contributions sees DC as routines that are simple in nature and more as best practices to
consider (EM). Thus, dynamic capabilities are more like behavioral patterns and
less idiosyncratic to the firm and will consequently not be able to satisfy the VRIO
condition for competitive advantage to emerge. These routines are  

On the other hand, a stream of contributions see dynamic capabilities as evolutionary and
more complex routines (Ref) or processes (ref) set to change underlying capabilities in
the face of changing conditions (TPS).  

CA RESULT OF DYNAIMC BUNDLE
to the socially complex and hard-to- imitate dynamic bundle of resources and capabili- ties (Peteraf et al., 2013) whose



Market path: simple rules guiding actions in markets for aquistion of assets. Not
themselves VRIO, but taken together with the assets they aquire more idiosyncratic to the
firm. These simple rules are routines and they corresponds to rapid changes and the rear
wheel. mor profound in dynamic environments
Eisenhardt and Sull (2001) credited its prowess in this arena to the simple rules that Cisco has developed to guide its acquisition process. Cisco

Evolutionary path: More complex rules and process working to change the operational
routines of the firm. These are themselves idiosyncratic and VRIO (vrio stems from the DC
level and hence generates path dependent routines at theOR level. Corresponds to more
gradual change. the front crank / more gradual environments. More complex processes need
to work thorugh a mediation (OR). More simple rules works more directly (unobserved)

On the notion of observability we can observe changes in underlying routines (OR) whereas
it is harder to observe routines generating opportunities in the market (dircet route)


Eisenhardt and Sull (2001) credited its prowess in this arena to the simple rules that
Cisco has developed to guide its acquisition process. Cisco

As an answer to the call for more nuanced theorizing on the differing role of dynamic
capabilities in shaping competitive advantage, and their contingencies
\citep{Peteraf2013}, \cite{DiStefano2014} propose theoretical model distinguishing between
rapid and routinized changes on the one hand, and slower, less routinized changes on the
other. Using the "organizational drive train" metaphor


The nature of DC: 
The fundamental difference between these two conceptions is related to the degree of observabil- ity. Action that is latent cannot be observed until called into use, while constituent elements have a more concrete and observable form (Helfat, Finkel- stein, Mitchell, Peteraf, Singh, & Winter, 2007). This has implications for the empirical identifica- tion of dynamic capabilities and suggests some of the challenges involved

DC as enabling device (latent) and as process (observable THROUGH changes in OR). 



%Latent action / process vs routine - can be both??
 %% Interestingly, the distinction be- tween dynamic capabilities as an ability versus dy- namic capabilities as a process dates back to the two seminal manuscripts, with TPS advocating the former position and EM supporting the latter. 

DRIVE TRAIN EXPLAINED


%% Given the extent and import of these differences
%% in the seminal views, how then can we bring them together fruitfully? Our answer is that we can do so by broadening our perspective to see the two views as each focused on a different part of a larger, interconnected, and more fully dynamic system.

We claoim, contrary to Di Stefano, that the two mechanisms in the drive train yields
differing results. The simple rules tend to only yield CA when they can operate in factor
markets to capture unique compistiosn of resources - ie.e their CA stems from their
ability to systematiclally find resrources in factor markets needed to adapt to
change. these resoures nor the routines themselves are sufficiently VRIO, but taken
together they form a powerful lever for change (increaseing or decreasing power from the
crankset).



OTHER METAPHOR: HYBRID CAR
Running electric motor in the gradual environment building speed continuosly (changing
rountines and hence moving more like a process.


ROLE OF DYN:
Last, our conception of an organizational drive-
train also suggests a resolution to the debate over whether dynamic capabilities in high-velocity en- vironments can go beyond managing change to also provide a sustainable competitive advantage. In this respect, we note that even if specific simple rules are unstable and ephemeral, the system as a whole is not. Moreover, the interconnectedness, reliance on tacit knowledge, and complexity of a system involving a variety of moving parts suggest that the real source of sustainable competitive ad- vantage for an enterprise is the difficulty of imitat- ing and substituting for the entire dynamic bundle (Peterafthat the system represents.


Previous work has highlighted the role of higher-order capabilities in a way that is
consistent with the idea of a dynamic bundle \citep{Schilke2014a}. This concept of higher-r
The view of the organiza- tional drivetrain is also consistent with works that tend to conceive dynamic capabilities as higher- order capabilities because it helps with envisioning different types of linkages between higher-order capabilities and organizational routines (see Kahl, 2014 and Schilke, 2014).
