Extant literature on dynamic capabilities are increasingly focusing on how firm resources
are changed by dynamic capabilities as a response to changes in the environment
\citep{Schilke2018,Eriksson2014}, thus placing firm capabilities as a mediator between
dynamic capabilities and competetive advantage \citep{Helfat2007}. This notion of
'orchestration' of firm resources, assets and capabilities has indeed been emphasized in
the extant literature \citep{Sirmon2011,Helfat2015}. 





The conventional mechanism explored in the extant literature is related to changes in the
firm`s resource base \citep{Eriksson2014,Schilke2018}. This mechanism is also included in
the dominating definition of dynamic capabilities found in \citep{Helfat2007}: 'A dynamic
capability is the capacity of an organization to purposefully create, extend or modify its
resource base' (p. 4). This definition opens up for a variety of components
included in the term 'resource base', a point we discuss later in this paper. We coin this
expression of dynamic capabilities as the {\bf conductor} envisioning a conductor
orchestrating firm capabilities.

This notion of a {\bf conductor} albeit very important to the development of dynamic
capabilities literature, seems to leave out a considerable part of a firm`s 'resource
base', namely those resources that are tied to how the employees of the firm sees their
mission, internalize information and effectuates change. These traits of the organization
is very much related to concepts of 'mindset' \citep{Dweck2016}, 'employee cognition'
\citep{Lakoff1987,Gavetti2012} 'organizational genes' \citep{Nelson1982} and
'sense-making' \citep{Weick1995} to name but a few. However, in relations with dynamic
capabilities, one promising avenue for exploring mechanisms, beyond the conventional ones
described above, makes a clear distinction between \emph{behavioral} and
\emph{non-behavioral} objects of change \citep{Verona2011}. These non-behavioral objects
of change contain psychological objects such as cognitive frames, motivation and identity
that are important for strategy formation and implementation \citep{Vince2011,Gavetti2012}
as they 'concerns how decisions or actions are shaped, subverted and transformed by
emotions, and (..) determine the "way we do things here"' \citep[p. 338]{Vince2011}. In
this respect they work as firm resources in generating outcomes such as performance and
competitive advantage. This notion is popularly perhaps expressed most clearly by Peter
Drucker`s famous quote about how 'culture eats strategy for lunch'. \footnote{The debate
  on whether Drucker actually uttered these words are still in the open, but it was
  attributed to him by Mark Fields in 2006} Moreover, these non-behavioral objects of
change are themselves far from static and are found to change over time as responses to
external change \citep{Zollo2016} thus making them a natural part of the mechanisms
underpinning how dynamic capabilities work \citep{Schilke2018,Verona2011}.  Still, far less
is know about the non-behavioral objects of change and how they work as a mechanism in the
dynamic capabilities theory.We coin {\bf coach} to capture the enabling role dynamic
capabilities plays through \emph{strategic cognition} as
juxtaposed to the more directional role played as {\bf conductor} of \emph{operational capabilities}.



